\documentclass{beamer}
\usetheme{CambridgeUS}

\usepackage{minted}
\usepackage{xcolor}
\usepackage{fontspec}
\usepackage{xunicode}
\usepackage{xltxtra}
\usepackage{xecyr}
\usepackage{hyperref}

\setmainfont[Mapping=tex-text]{CMU Serif}
\setsansfont[Mapping=tex-text]{CMU Sans Serif}                %% задаёт шрифт без засечек
\setmonofont[Mapping=tex-text]{CMU Typewriter Text}           %% задаёт моноширинный шрифт
\usepackage{polyglossia}

\setdefaultlanguage{russian}
\usepackage{amssymb}
\usepackage{graphicx}
\usepackage{listings}

\usepackage{faktor}
\usepackage{stmaryrd}

\newcommand{\alphaeq}{=_{\alpha}}
\newcommand{\betaeq}{=_{\beta}}

\begin{document}
\title[Формализация $\lambda$-исчисления]{Реализация сравнения различных представлений лямбда-термов}
\subtitle{предварительные результаты}
\author[Жаворонков Э.А.]{студент: Жаворонков Эдгар\\{\footnotesize\textcolor{gray}{научный руководитель: В.И. Исаев}}}
\date{11/05/2017}
\institute{СПб АУ НОЦНТ РАН}

\frame{\titlepage}

\begin{frame}\frametitle{Введение. $\lambda$-термы и связанные определения}
    \begin{enumerate}
        \item Термы $\lambda$-исчисления строятся индуктивно, путем применения и создания анонимных функций.
        \item Пусть $V = \{x, y, \dots\}$ -- счетное множество переменных. Тогда множество $\lambda$-термов в абстрактном синтаксисе определяется как $\Lambda ::= V | \Lambda\ \Lambda | \lambda V. \Lambda$
        \item Примеры термов -- $\lambda f g x. f\ x\ (g\ x)$ или $(\lambda x. f\ (x\ x))(\lambda x. f\ (x\ x))$.
    \end{enumerate}
\end{frame}


\begin{frame}\frametitle{Введение. $\lambda$-термы и связанные определения(2)}
    \begin{enumerate}
        \item Переменная, которая стоит после $\lambda$ называется \textit{связанной}. Соответственно, до абстракции она была \textit{свободной}.
        \item В первом примере $f$ --- связана, а во втором --- свободна.
        \item Имена связанных переменных не важны. $\lambda f g x.f\ x\ (g\ x)$ при применении к аргументам ведет себя так же, как  и $\lambda x y z. x\ z\ (y\ z)$.
    \end{enumerate}
\end{frame}


\begin{frame}\frametitle{Введение. $\lambda$-термы и связанные определения(3)}
    \begin{enumerate}
        \item На $\Lambda$ можно задать отношение $\alpha$-эквивалентности, которое и говорит о том, что имена формальных параметров лямбд не важны.
        \item Для $\lambda$-термов определена операция подстановки $M[x \mapsto N]$. Она описывает, что происходит когда абстракция применяется к аргументу.
        \item Мы записывали термы, используя именованные переменные. Но это не единственный способ записи. Помимо него есть еще неименованное представление и монадическое. Но о них позже.
    \end{enumerate}
\end{frame}

% Здесь ты должен ясно ответить на вопрос "Зачем это надо"
\begin{frame}\frametitle{Мотивация}
    \begin{enumerate}
        \item Соответствие Карри-Говарда говорит нам о том, что между логикой и теорией типов есть прямая связь.
        \item Основная мысль состоит в том, что для доказательства утверждений мы можем писать $\lambda$-термы(читай -- программы).
        \item Чем больше логических связок у нас есть, тем более мощные теории типов приходится использовать.
    \end{enumerate}
\end{frame}


\begin{frame}\frametitle{Мотивация(2)}
    \begin{enumerate}
        \item Теории с зависимыми типами позволяют доказывать утверждения в которых есть кванторы($\forall x P(x)$ или $\exists x P(x)$).
        \item Известные примеры таких теорий -- исчисление конструкций или интуиционистская теория типов Мартин-Лёфа.
        \item Их расширения лежат в основе систем автоматического доказательства теорем Coq и Agda.
    \end{enumerate}
\end{frame}


\begin{frame}\frametitle{Мотивация(3)}
    \begin{enumerate}
        \item Для доказательств иногда полезна функциональная экстенсиональность.
        \item В Agda она постулируется, притом сами разработчики языка не рекомендуют использовать постулаты по очевидным причинам.
        \item Хочется доказывать свойства термов, не используя "плохие" конструкции. Есть ли такая теория типов(такой язык)?
        \item <2-> Да, есть. Он называется Vclang\only<2->\footnote{https://github.com/valis/vclang} и основан на гомотопической теории типов с типом интервала. 
    \end{enumerate}
\end{frame}


\begin{frame}\frametitle{Цель и задачи}
    \textbf{Цель работы} -- формализовать различные представления $\lambda$-термов и их свойства.
    
    \bigskip
    
    \textbf{Задачи:}
    \begin{enumerate}
        \item Описать типы данных для термов.
        \item Описать свойства различных представлений($\alpha$-эквивалентность, операция подстановки etc.).
        \item Показать, что различные представления эквивалентны между собой.
        \item Для каждого представления доказать соответствующие свойства.
    \end{enumerate}
\end{frame}

% плохой слайд
\begin{frame}\frametitle{Существующие решения}
    \begin{enumerate}
        \item Задача формализации $\lambda$-исчисления довольно популярна.
        \item Одно из решений -- A Formalization of Typed and Untyped λ-Calculi in SSReflect-Coq and Agda2. \footnote{https://github.com/pi8027/lambda-calculus}
        \item В нем, как и в аналогичных описывается только одно представление.
        \item Кроме того, нам не очень нравится Agda как язык реализации. В нем нет конструкций, которыми нам хотелось бы воспользоваться(фактор-типы).
    \end{enumerate}
\end{frame}

% еще по слайду наверное надо добавить на каждое представление
\begin{frame}[fragile=singleslide]\frametitle{Решение. Именованное представление}
    Тип данных для термов:
        \begin{figure}[H]
            \center
            \begin{minted}{haskell}
            data Term =
                Var Name 
                | App Term Term 
                | Lam Name Term
            \end{minted}
        \end{figure}
    \begin{enumerate}
        \item Нужно разрешимое равенство на $Name$.
        \item $\alpha$-эквивалентность определяется индукцией по структуре терма плюс некоторый трюк в случае абстракции.
        \item Подстановка определяется аналогично индукцией, но для абстракций снова необходим трюк!
    \end{enumerate}
\end{frame}


\begin{frame}[fragile=singleslide]\frametitle{Решение. Именованное представление(2)}
    \begin{enumerate}
        \item Все утверждения о подстановке рассматривают термы с точностью до $\alpha$-эквивалентности.
        \item Свойства подстановки формулируются следующим образом:
            \begin{align*}
                &x[x \mapsto t] \alphaeq t\\
                &t[x \mapsto x] \alphaeq t\\
                &t[x \mapsto N][y \mapsto M] \alphaeq t[y \mapsto M][x \mapsto N[y \mapsto M]] (x \notin FV(M))
            \end{align*}
        \item Доказываются очень нудной индукцией по терму $t$.
    \end{enumerate}
\end{frame}


\begin{frame}[fragile=singleslide]\frametitle{Решение. Неименованное представление}
    Термы:
    \begin{figure}[H]
            \center
            \begin{minted}{haskell}
            data Term (n : Nat) =
                Var (i : Fin n) 
                | App (Term n) (Term n)
                | Lam (Term (suc n))
            \end{minted}
        \end{figure}
    \begin{enumerate}
        \item Подстановка в таком представлении -- полная(во все переменные)
        \item Так как нет имен переменных, то и определяется она намного проще.
    \end{enumerate}
\end{frame}


\begin{frame}[fragile=singleslide]\frametitle{Решение. Монадическое представление}
    Термы:
    \begin{figure}[H]
            \center
            \begin{minted}{haskell}
            data Term (V : Set) =
                Var V 
                | App (Term V) (Term V)
                | Lam (Term (Maybe V))
            \end{minted}
        \end{figure}
    \begin{enumerate}
        \item Нетрудно заметить, что это функтор.
        \item Чуть менее очевидно, но это монада!
        \item Монадные законы в точности описывают свойства подстановки.
        \item Для их доказательств удобно использовать функциональную экстенсиональность, которая в нашей теории конструируется, а не постулируется.
    \end{enumerate}
\end{frame}


\begin{frame}\frametitle{Результаты}
    \begin{enumerate}
        \item Написаны преобразования:
            \begin{enumerate}
                \item Из именованных термов в неименованные.
                \item Из неименованных в монадические.
                \item И обратно.
            \end{enumerate}
        \item Доказано, что преобразования -- биекции.
        \item Для именованных термов описана $\alpha$-эквивалентность.
        \item Для каждого представления определена операция подстановки.
        \item Для каждого представления доказаны свойства термов:
            \begin{enumerate}  
                \item Преобразование именованного терма в неименованный уважает $\alpha$-эквивалентность.
                \item Свойства операции подстановки.  
            \end{enumerate}
    \end{enumerate}
\end{frame}

\begin{frame}{fin}
    \begin{center} 
        \Huge Спасибо за внимание! \\ Вопросы? 
    \end{center} 
    
    \begin{block}{Github repo:}
        \footnotesize{https://github.com/edgarzhavoronkov/vclang-lib/tree/lambda\_calculus/test/LC}
    \end{block}
\end{frame} 


\end{document}