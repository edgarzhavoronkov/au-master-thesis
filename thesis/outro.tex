\section*{Заключение}

В ходе выполнения работы была рассмотрена задача доказательства равенства между различными представлениями лямбда-термов. Для этого мы описали различные представления термов, определили фундаментальную операцию подстановки и описали её свойства. Помимо этого, мы формально доказали равенство описанных ранее представлений с помощью системы автоматического доказательства теорем, что отличает эту работу, от существующих работ на смежную тематику.

Мы формализовали неименованное, монадическое и часть именованного представления термов. Мы выяснили, что с неименованным и монадическим представлением очень удобно работать в том смысле, что формальные определения без особых усилий перекладываются на язык доказателя теорем. Для этого мы использовали экспериментальный язык, построенный на новой теории типов, которая, в частности, позволяет описывать некоторые конструкции гораздо более удобным образом, нежели, например, \textbf{Agda}.

Однако, работа не лишена недостатков. Технические проблемы в самом языке и сложность именованного представления для формализации привели к тому, что свойство ассоциативности для операции подстановки осталось неформализованным. В качестве решения этой проблемы можно использовать идеи, описанные в теории номинальных множеств. Мы оставим эти идеи, как направление дальнейшего развития работы.

Исходный код опубликован в публичном Git-репозитории и доступен по ссылке \url{https://github.com/edgarzhavoronkov/vclang-lib/tree/lambda_calculus/test/LC}
