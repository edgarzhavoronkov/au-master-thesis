\section*{Заключение}

В ходе выполнения работы была рассмотрена задача доказательства равенства между различными представлениями лямбда-термов. Для ее решения мы использовали экспериментальный язык, построенный на новой теории типов, которая, в частности, позволяет описывать некоторые конструкции гораздо более удобным образом, нежели, например, \textbf{Agda}.

С помощью этого языка мы реализовали различные представления термов: именованное, неименованное и монадическое. Мы реализовали фундаментальную операцию подстановки и сформулировали её свойства -- унитальность и ассоциативность. мы полностью доказали эти свойства для неименованного и монадических представлений. Для именованного представления небольшая часть доказательств не была доведена до конца в связи с техническими проблемами в самом языке. Помимо этого, мы построили преобразования между этими представлениями и доказали, что они взаимно-обратны, тем самым, показав равенство. Однако следует сделать замечание, что так как для именованного представления термов нам нужен еще и контекст, мы показали эквивалентность между парами из термов и контекстов и неименованными термами.

Один из вариантов развития работы, который может помочь в решении проблем, возникших при формализации именованного представления -- это номинальные множества. Мы оставим его, как направление развития.

Исходный код опубликован в публичном Git-репозитории и доступен по ссылке \url{https://github.com/edgarzhavoronkov/vclang-lib/tree/lambda_calculus/test/LC}
