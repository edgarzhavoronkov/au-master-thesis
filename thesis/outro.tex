\section*{Заключение}

% Снова примерный план того, что хочется увидеть:
% \begin{enumerate}
%   \item Что ты содержательного сделал в ходе выполнения работы
%   \item Чем твое решение лучше, чем осталльные
%   \item Можешь вкратце сказать, что в качестве языка реализации использовался vclang -- экспериментальный инструмент
%   \item Какие недостатки есть в твоем решении
%   \item Какое дальнейшее развитие может быть у твоей работы
%   \item Ссылку на исходники
% \end{enumerate}

В ходе выполнения работы была рассмотрена с новой точки зрения задача формализации лямбда-исчисления. Мы описали различные представления лямбда-термов, определили фундаментальную операцию подстановки и описали её свойства.

В отличии от других решений этой задачи, мы установили соответствие между различными представлениями лямбда-термов. Кроме того, мы предложили формализацию монадического представления термов. Мы использовали экспериментальный язык, построенный на новой теории типов, которая позволяет описывать фактор-множества гораздо более удобным образом, нежели, например, \textbf{Agda}.

Однако, работа не лишена недостатков. Технические проблемы в самом языке и сложность именованного представления для формализации привели к тому, что свойство ассоциативности для операции подстановки осталось неформализованным. В качестве решения этой проблемы можно использовать идеи, описанные в теории номинальных множеств. Мы оставим эти идеи, как направление дальнейшего развития работы.

Исходный код опубликован в публичном Git-репозитории и доступен по ссылке \url{https://github.com/edgarzhavoronkov/vclang-lib/tree/lambda_calculus/test/LC}
