\subsection{Существующие решения}

Стоит отметить, что задача формализации $\lambda$-исчисления довольно популярна, в связи с чем существует довольно много её решений. Один из примеров -- \cite{lambdaForm}, в котором, в частности, формализовано чистое $\lambda$-исчисление. Для него, как и для работ~\cite{shankar1988mechanical, altenkirch1993formalization, barras1996coq, nipkow1996more, huet1994residual} характерно использование неименованного представления термов через индексы Де Брауна. Работы~\cite{mckinna1999some, coquand1996algorithm, gabbay1999new, gordon1996five, sato1983theory, stoughton1988substitution} отличаются, от вышеперечисленных тем, что авторы формализуют именованное представление термов. Есть работа~\cite{altenkirch1999monadic}, в которой формализуется монадическое представление термов.

Вышеперечисленные работы объединяет то, что авторы рассматривают только одно представление термов(именованное, неименованное или монадическое), не устанавливая между ними никакого соответствия. Кроме того, некоторые работы формализуют типизированные вариации лямбда-исчисления. Есть, однако, работа~\cite{berghofer2007head}, в которой реализовано сравнение именованного и неименованного представлений, но в том смысле, что авторы реализовали оба представления термов и сделали выводы о том, какое из оказалось удобнее для реализации.

% TODO: подробнее, здесь надо раскрыть новизну твоей работы
Мы, в свою очередь, хотим не просто формализовать различные представления чистого лямбда-исчисления, но и установить между ними соответствие. Именно, мы хотим доказать, что различные представления термов равны между собой в том смысле, что равны типы, соответствующие этим представлениям.

Более того, наше решение будет использовать отличную от других теорию типов. В разделе~\ref{sec:impl} мы подробнее опишем язык, построенный на основе этой теории и опишем, чем она отличается от других известных теорий. Здесь же мы просто скажем, что в этом языке есть конструкции, которых нет в других языках программирования с зависимыми типами, либо ими сложно пользоваться(речь идет, например, о фактор-типах).
