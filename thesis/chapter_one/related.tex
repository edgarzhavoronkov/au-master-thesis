\subsection{Существующие решения}

Стоит отметить, что задача формализации $\lambda$-исчисления довольно популярна, в связи с чем существует довольно много её решений. Один из примеров -- \cite{lambdaForm}, в котором, в частности, формализовано чистое $\lambda$-исчисление. Для него, как и для работ~\cite{shankar1988mechanical, altenkirch1993formalization, barras1996coq, nipkow1996more, huet1994residual} характерно использование неименованного представления термов через индексы Де Брауна. Работы~\cite{mckinna1999some, coquand1996algorithm, gabbay1999new, gordon1996five, sato1983theory, stoughton1988substitution} отличаются, от вышеперечисленных тем, что авторы формализуют именованное представление термов.

Вышеперечисленные работы объединяет то, что авторы рассматривают только одно представление термов(именованное или неименованное), не устанавливая между ними никакого соответствия. Кроме того, некоторые работы формализуют типизированные вариации лямбда-исчисления. Мы, в свою очередь, хотим формализовать \textit{различные представления} чистого лямбда-исчисления и установить между ними соответствие. Помимо этого, нам неизвестны попытки формализовать монадическое представление термов.

Более того, наше решение будет использовать отличную от других теорию типов. В разделе~\ref{sec:impl} мы подробнее опишем язык, построенный на основе этой теории и опишем, чем она отличается от других известных теорий. Здесь же мы просто скажем, что в этом языке есть конструкции, которых нет в других языках программирования с зависимыми типами, либо ими сложно пользоваться(речь идет, например, о фактор-типах).
