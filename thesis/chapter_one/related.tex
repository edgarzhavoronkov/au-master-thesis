\subsection{Существующие решения}

Стоит отметить, что задача формализации $\lambda$-исчисления довольно популярна, в связи с чем существует довольно много её решений. Один из примеров -- \cite{lambdaForm}, в котором, в частности, формализовано чистое $\lambda$-исчисление. Для него, как и для работ~\cite{shankar1988mechanical, altenkirch1993formalization, barras1996coq, nipkow1996more, huet1994residual} характерно использование неименованного представления термов через индексы Де Брауна. Работы~\cite{mckinna1999some, coquand1996algorithm, gabbay1999new, gordon1996five, sato1983theory, stoughton1988substitution} отличаются, от вышеперечисленных тем, что авторы формализуют именованное представление термов. Есть работа~\cite{altenkirch1999monadic}, в которой формализуется монадическое представление термов.

Вышеперечисленные работы объединяет то, что авторы рассматривают только одно представление термов(именованное, неименованное или монадическое), не устанавливая между ними никакого соответствия. Кроме того, некоторые работы формализуют типизированные вариации лямбда-исчисления. Есть, однако, работа~\cite{berghofer2007head}, в которой реализовано сравнение именованного и неименованного представлений, но в том смысле, что авторы реализовали оба представления термов и сделали выводы о том, какое из оказалось удобнее для реализации.

Стоит отметить и работу~\cite{tarau2015logic}, в которой средствами логического программирования на языке \textbf{Prolog} построены биекции между именованным и неименованным представлениями лямбда-термов, хоть и не совсем понятно, каким именно образом авторы показали, что эти соответствия взаимно-обратны.

Мы, в свою очередь, хотим не просто формализовать различные представления чистого лямбда-исчисления, но и установить между ними соответствие. Именно, мы хотим доказать, что различные представления термов равны между собой в том смысле, что равны типы, соответствующие этим представлениям. С этим связан тонкий момент, потому что именованные термы рассматриваются с точностью до $\alpha$-эквивалентности(формально это понятие будет введено далее в разделе~\ref{sec:named}). То есть мы рассматриваем не просто множество термов, но множество термов, в котором $\alpha$-эквивалентные термы отождествлены, так называемое фактор-множество(формальное определение этого понятия, опять же будет дано в разделе~\ref{sec:named}).

В языках программирования существуют различные конструкции, которые в том или ином роде моделируют фактор-множества. Самый простой способ сделать это -- переопределить операцию сравнения на типе. В системе автоматического доказательства теорем \textbf{Agda} аналогом фактор-множеств являются так называемые \textit{сетоиды} -- множества с введенным на них отношением эквивалентности. Технически, они устроены, как записи из подлежащего типа и отношения.

В нашем случае, можно обойтись и формализовывать именованное представление без фактор-типов и использовать сетоиды, но тогда вместо доказательстве равенства типов(эквивалентности между различными представлениями) нам придется вводить и доказывать отношение эквивалентности сетоидов. Отношение равенства мы считаем более предпочтительным по многим причинам, в частности, для него верен принцип Лейбница(принцип подстановки эквивалентных).

Новизна нашего решения заключается в том, что оно будет использовать отличную от других теорию типов, которая позволяет удобным образом конструировать фактор-типы. В разделе~\ref{sec:impl} мы подробнее опишем язык, построенный на основе этой теории. Мы сравним его с известными системами автоматического доказательства теорем и выделим его достоинства, которые привели нас к выбору именно этого языка.
