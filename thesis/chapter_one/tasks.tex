\subsection{Постановка задачи}

Целью работы является формализация различных представлений $\lambda$-термов. Для достижения этой цели необходимо решить следующие задачи:

\begin{enumerate}
  \item Определить интересующие нас представления лямбда-термов:
    \begin{enumerate}
      \item Именованное
      \item Неименованное
      \item Монадическое
      \item Представление с помощью фактор-множества
    \end{enumerate}
  \item Для каждого представления определить операцию подстановки
  \item Для каждого представления доказать свойства операции подстановки:
    \begin{enumerate}
      \item Унитальность(наличие правой и левой единицы)
      \item Ассоциативность
    \end{enumerate}
  \item Установить соответствие между описанными в первом пункте представлениями
\end{enumerate}

Операция подстановки является фундаментальной операцией над термами. С помощью нее вводятся отношения, которые наделяют $\lambda$-исчисление вычислительной семантикой, например $\beta$-редукция. Унитальность и ассоциативность -- её основные свойства, поэтому мы хотим формализовать именно эту операцию.
