\subsection{Постановка задачи}
% Переписать так, чтобы задачи в текущей постановке еще не были никем до тебя решены. Должна быть очевидна новизна.
Целью работы является доказательство равенства между различными представлениями $\lambda$-термов. Для это мы планируем решить следующие задачи:

\begin{enumerate}
  \item Исследовать существующие решения задачи формализации лямбда-исчисления с целью доказательства равенства между различными представлениями термов.
  \item Формализовать интересующие нас представления лямбда-термов с помощью языка \textbf{Vclang}:
    \begin{enumerate}
      \item Именованное
      \item Неименованное
      \item Монадическое
    \end{enumerate}
  \item Для каждого представления реализовать операцию подстановки
  \item Для каждого представления формализовать свойства операции подстановки:
    \begin{enumerate}
      \item Унитальность(наличие правой и левой единицы)
      \item Ассоциативность
    \end{enumerate}
  \item С помощью языка \textbf{Vclang} доказать, что описанные в первом пункте представления равны между собой
\end{enumerate}

Операция подстановки является фундаментальной операцией над термами. С её помощью вводятся отношения(например $\beta$-редукция), которые позволяют не просто записывать $\lambda$-выражения, но и сокращать их, то есть вычислять. Унитальность и ассоциативность -- её основные свойства, поэтому мы хотим формализовать именно эту операцию.
