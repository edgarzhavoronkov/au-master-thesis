\subsection{Мотивация}

Как уже упоминалось в введении, языки с зависимыми типами являют собой полноценную логику, позволяя использовать их как инструмент для формализации математики.

Это возможно благодаря соответствию Карри-Говарда, которое устанавливает связь между логикой и теорией типов. То есть, логической связке соответствует конструкция в теории типов, а логическому утверждению -- тип. Доказательству того факта, что утверждение истинно, соответствует тогда доказательство того факта, что соответствующий этому утверждению тип населен. Иначе говоря, мы можем предъявить терм соответствующего типа, чтобы доказать исходное утверждение. Сведем для наглядности некоторые соответствия в таблицу:

\begin{table}[H]
  \centering
  \begin{tabular}{| c | c |}
    \hline
    Высказывание $A$: & Тип $A$: \\
    \hline
    Тождественная истинна & $\top$(единичный тип) \\
    \hline
    Тождественная ложь & $\bot$(пустой тип без обитателей) \\
    \hline
    $\lnot A$(отрицание) & $A \to \bot$ \\
    \hline
    $A \land B$(конъюнкция) & $A \times B$(тип-произведение) \\
    \hline
    $A \lor B$(дизъюнкция) & $A \coprod B$(тип-сумма) \\
    \hline
    $A \to B$(импликация) & $A \to B$(тип функций из $A$ в $B$) \\
    \hline
    $\exists x.P(x)$ & $\Sigma (x : A) (P a)$(тип зависимых пар) \\
    \hline
    $\forall x.P(x)$ & $\Pi (x : A) \to (P a)$(тип зависимых функций) \\
    \hline
  \end{tabular}
  \caption{Соответствия высказываний в логике и конструкций в теории типов}
\end{table}
Чем больше логических связок мы хотим использовать, тем более мощные теории типов  нам придется использовать, чтобы доказывать эти утверждения. Так, например, если нам потребуется доказывать формулы пропозициональной логики, то мы можем обойтись просто типизированным $\lambda$-исчислением. Если нам понадобятся кванторы в формулах, то на помощь приходят теории с зависимыми типами.

Существует две известные теории с зависимыми типами -- исчисление конструкций(Calculus Of Constructions), представленное Тьерри Коканом в \cite{coquand1988calculus} и интуиционистская теория типов Мартин-Лёфа(Martin-L{\"o}f Type Theory), описанная в \cite{martin1975intuitionistic}. Их расширения лежат в основе таких систем интерактивного доказательства теорем как \textbf{Coq} и \textbf{Agda} соответственно. Нам сейчас не очень важно, чем именно отличаются эти две теории, поэтому мы не будем заострять на этом внимание.

Отметим, что для построения доказательств иногда бывает полезна функциональная экстенсиональность. Это конструкция, которая по двум функциям $f, g : X \to Y$ и доказательству $\forall x (f x \equiv g x)$ возвращает нам доказательство $f \equiv g$. В \textbf{Agda}, например, функциональной экстенсиональности нет -- её можно постулировать, однако сами авторы не рекомендуют использовать постулаты по той простой причине, что их использование позволяет, например, предъявить элемент пустого множества.

Нам хотелось бы строить доказательства, не используя таких <<неугодных>> с точки зрения языка конструкций. Незамедлительно возникает вопрос, есть такая теория типов(а еще лучше -- язык), в которой функциональная экстенсиональность была бы сконструирована должным образом, а не постулирована? Ответ -- да, есть.

% Написать про Vclang, Валерину теорию, дать ссылку(гитхаб?)
