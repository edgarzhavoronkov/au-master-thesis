\subsection{Краткое введение в \texorpdfstring{$\lambda$}{лямбда}-исчисление}

Лямбда-исчисление -- это формальная система, впервые описанная Алонзо Черчём в статье \cite{church1936unsolvable} с целью формализации понятия вычислимости. Изначально безтиповое, но с течением времени появилось множество типизированных вариаций, образующих так называемый $\lambda$-куб, описанный Барендрегтом в \cite{barendregt1993lambda}. Далее говоря <<$\lambda$-исчисление>>, мы будем иметь в виду чистое $\lambda$-исчисление.

Термы $\lambda$-исчисления($\lambda$-термы) конструируются из переменных путем применения друг к другу или создания анонимных функций.

Формально, пусть $\mathcal{V}=\{x,y,z,\dots\}$ -- счетное множество переменных. Договоримся обозначать переменные прописными буквами, а произвольные термы -- заглавными. Тогда множество $\lambda$-термов $\Lambda$ определяется индуктивно, согласно следующим правилам:
\begin{align*}
  v \in \mathcal{V} &\Rightarrow v \in \Lambda \\
  M, N \in \Lambda &\Rightarrow M N \in \Lambda \\
  v \in \mathcal{V}, M \in \Lambda &\Rightarrow \lambda v.M \in \Lambda
\end{align*}

Нотация аппликации $M N$ обозначает применение функции $M$ ко входу $N$. Заметим, что так как здесь не вводится никаких правил типизации, то ничто не мешает нам применить терм к самому себе(i.e $F F$). Нотация абстракции $\lambda x.M$, в свою очередь, обозначает создание анонимной функции от переменной $x$, которая сопоставляет конкретному значению $x$ выражение $M[x]$. Здесь заметим, что терм $M$ вовсе не обязан содержать в себе переменную $x$, в таком случае мы считаем абстракцию $\lambda x.M$ константной функцией.

% примеры термов напиши

Переменная $x$ после абстракции $\lambda x.M$ называется \textit{связанной}. Соответственно, до абстракции она была \textit{свободной}. Формально, множества $FV(T)$ свободных и $BV(T)$ связанных переменных терма $T$ определяются индуктивно следующим образом:
\begin{align*}
  FV(x) &= \{x\} \\
  FV(M N) &= FV(M) \cup FV(N) \\
  FV(\lambda x. M) &= FV(M) \setminus \{x\} \\
  \\
  BV(x) &= \emptyset \\
  BV(M N) &= BV(M) \cup BV(N) \\
  BV(\lambda x. M) &= \{x\} \cup BV(M)
\end{align*}

Применение абстракции к некоторому аргументу $(\lambda x.M) N$ -- это \textit{подстановка} $M[x \mapsto N]$ терма $N$ вместо \textit{свободных} вхождений переменной $x$ в терме $M$. Формально, правила подстановки:
\begin{align*}
  x[x \mapsto N] &= N \\
  y[x \mapsto N] &= y, (x \neq y) \\
  (T S)[x \mapsto N] &= T[x \mapsto N] S[x \mapsto N] \\
  (\lambda x.T)[x \mapsto N] &= \lambda x.T \\
  (\lambda y.T)[x \mapsto N] &= \lambda y.T[x \mapsto N], (y \notin FV(N), x \neq y)
\end{align*}

Рассмотрим теперь два терма $\lambda x.x$ и $\lambda y.y$. Заметим, что при применении их к произвольному аргументу они ведут себя абсолютно одинаково. Неформально говоря, имена связанных переменных не играют для нас никакой роли. Формально, это отношение называется $\alpha$-эквивалентностью и определяется как минимальное отношение конгруэнтности, порожденное следующими правилами:

\begin{center}
  \AxiomC{$x \in \mathcal{V}$}
  \UnaryInfC{$x \alphaeq x$}
  \DisplayProof{}
\end{center}

\begin{center}
  \AxiomC{$M \alphaeq M'$}
  \AxiomC{$N \alphaeq N'$}
  \BinaryInfC{$M N \alphaeq M' N'$}
  \DisplayProof{}
\end{center}

\begin{center}
  \AxiomC{$M[x \mapsto y] \alphaeq M'$}
  \AxiomC{$y \notin FV(M)$}
  \BinaryInfC{$\lambda x. M \alphaeq \lambda y.M'$}
  \DisplayProof{}
\end{center}

Отметим, наконец, что мы записывали термы используя именованные переменные. Это не единственный вариант их представления. Во второй главе мы опишем еще два способа записи термов, отметив их достоинства и недостатки.
