\subsection{\texorpdfstring{$\lambda$}{Лямбда}-исчисление}
\label{sec:lambda}

\paragraph{Историческая справка}
Лямбда-исчисление -- это формальная система, придуманная в 30-ых годах прошлого века Алонзо Черчём(англ. Alonso Church)~\cite{church1936unsolvable} с целью анализа и формализации понятия вычислимости. В 60-ых годах Питером Ландином(англ. Peter Landin) была опубликована работа~\cite{landin1964mechanical}, в которой выдвигалась идея о том, что $\lambda$-исчисление может использоваться для моделирования различных выражений в языках программирования того времени, что в дальнейшем привело к развитию языков в стиле \textbf{ML}. С тех пор идеи $\lambda$-исчисления широко используются в мире функционального программирования.

\paragraph{Неформальное описание $\lambda$-термов}
Мы формально определим $\lambda$-термы во второй главе, здесь же мы просто скажем, что термы $\lambda$-исчисления рекурсивно конструируются из переменных с помощью всего двух операций -- применения функции к аргументу и создания анонимной функции. Наличие каких-либо констант здесь не предполагается. Несмотря на кажущуюся простоту, $\lambda$-исчисление является очень мощной формальной системой, в частности, Шейнфинкелем(нем. Moses Sch{\"o}nfinkel) и Карри(англ. Haskell Brooks Curry) в работах \cite{schonfinkel1924bausteine, curry1930grundlagen} введен в рассмотрение базис из двух термов(комбинаторов) $S = \lambda f g x. f x (g x)$ и $K = \lambda x y. x$, который примечателен тем, что обладает полнотой по Тьюрингу.

Отметим, что лямбда-термы допускают не единственный способ записи. В работе мы неоднократно будем употреблять термин <<представление термов>>. Под ним мы будем понимать способ, которым термы кодируются на каком-либо языке. Мотивация к появлению различных представлений $\lambda$-термов состоит в том, что являясь языком программирования, лямбда-исчисление само столкнулось с проблемой коллизии имен переменных. Различные представления термов по-разному решают эту проблему. Как следствие, какие-то представления удобнее для восприятия человеком и неформальных рассуждений <<на бумаге>>. Какие-то представления удобны для компьютерной реализации и используются как внутреннее представление в функциональных языках программирования или системах автоматического доказательства теорем, как например в \textbf{Agda}~\cite{norell2007towards}.
