\subsection{Соответствие Карри-Говарда}
Соответствие Карри-Говарда~\cite{howard1980formulae} устанавливает прямую связь между логикой и теорией типов. Логической связке соответствует конструкция в теории типов, а логическому утверждению -- тип. Доказательству того факта, что утверждение истинно, соответствует тогда доказательство того факта, что соответствующий этому утверждению тип населен. Иначе говоря, мы можем предъявить $\lambda$-терм соответствующего типа, чтобы доказать исходное утверждение. Для наглядности некоторые соответствия сведены в таблицу:

\begin{table}[H]
  \centering
  \begin{tabular}{| c | c |}
    \hline
    Высказывание $A$: & Тип $A$: \\
    \hline
    Истинно & Населен \\
    \hline
    Тождественная истина & $\top$(единичный тип) \\
    \hline
    Тождественная ложь & $\bot$(пустой тип без обитателей) \\
    \hline
    $\lnot A$(отрицание) & $A \to \bot$ \\
    \hline
    $A \land B$(конъюнкция) & $A \times B$(тип-произведение) \\
    \hline
    $A \lor B$(дизъюнкция) & $A \coprod B$(тип-сумма) \\
    \hline
    $A \to B$(импликация) & $A \to B$(тип функций из $A$ в $B$) \\
    \hline
    $\exists x.P(x)$ & $\Sigma (x : A) (P a)$(тип зависимых пар) \\
    \hline
    $\forall x.P(x)$ & $\Pi (x : A) \to (P a)$(тип зависимых функций) \\
    \hline
  \end{tabular}
  \caption{Соответствия высказываний в логике и конструкций в теории типов}
\end{table}

Чем больше логических связок мы хотим использовать, тем более мощные теории типов  нам придется использовать, чтобы доказывать эти утверждения. Так, например, если нам потребуется доказывать формулы пропозициональной логики, то мы можем обойтись просто типизированным $\lambda$-исчислением. Если нам понадобятся кванторы в формулах, то на помощь приходят теории с зависимыми типами. Таким образом, благодаря соответствию Карри-Говарда, процесс доказательства утверждений становится похож на программирование, следовательно, задача проверки корректности доказательств сводится к задаче проверки типов.

Существует две известные теории с зависимыми типами -- исчисление конструкций(Calculus Of Constructions), представленное Тьерри Коканом в \cite{coquand1988calculus} и интуиционистская теория типов Мартин-Лёфа(Martin-L{\"o}f Type Theory), описанная в \cite{martin1975intuitionistic}. Их расширения лежат в основе таких систем интерактивного доказательства теорем(языков с зависимыми типами) как \textbf{Coq} и \textbf{Agda} соответственно. Обычно эти языки используют для верификации программ, но так как они являются полноценной логикой, то с их помощью можно формулировать и доказывать математические утверждения. Говоря <<формализация>>, мы будем иметь в виду именно это.
