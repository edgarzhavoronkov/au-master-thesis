\subsection{Именованное представление термов}
\label{sec:named}

Термы $\lambda$-исчисления($\lambda$-термы) в именованном представлении конструируются из переменных путем применения друг к другу или создания анонимных функций.

Формально, пусть $\mathcal{V}=\{x,y,z,\dots\}$ -- счетное множество переменных. Договоримся обозначать переменные прописными буквами, а произвольные термы -- заглавными. Тогда множество $\lambda$-термов $\Lambda$ определяется индуктивно, согласно следующим правилам:
\begin{center}
  \AxiomC{$v \in \mathcal{V}$}
  \UnaryInfC{$v \in \Lambda$}
  \DisplayProof{}
\end{center}

\begin{center}
  \AxiomC{$M \in \Lambda$}
  \AxiomC{$N \in \Lambda$}
  \BinaryInfC{$M N \in \Lambda$}
  \DisplayProof{}
\end{center}

\begin{center}
  \AxiomC{$M \in \Lambda$}
  \AxiomC{$v \in \mathcal{V}$}
  \BinaryInfC{$\lambda v.M \in \Lambda$}
  \DisplayProof{}
\end{center}

Нотация аппликации $M N$ обозначает применение функции $M$ ко входу $N$. Заметим, что так как здесь не вводится никаких правил типизации, то ничто не мешает нам применить терм к самому себе(i.e $F F$). Нотация абстракции $\lambda x.M$, в свою очередь, обозначает создание анонимной функции от переменной $x$, которая сопоставляет конкретному значению $x$ выражение $M[x]$. Здесь заметим, что терм $M$ вовсе не обязан содержать в себе переменную $x$, в таком случае мы считаем абстракцию $\lambda x.M$ константной функцией.

Некоторые примеры термов:
\begin{gather*}
   \lambda x.x \\
   \lambda x y.x \\
   (\lambda x.f (x x)) (\lambda x.f (x x))
\end{gather*}

Переменная $x$ после абстракции $\lambda x.M$ называется \textit{связанной}. Соответственно, до абстракции она была \textit{свободной}. Формально, множества $FV(T)$ свободных и $BV(T)$ связанных переменных терма $T$ определяются индуктивно следующим образом:
\begin{align*}
  FV(x) &= \{x\} \\
  FV(M N) &= FV(M) \cup FV(N) \\
  FV(\lambda x. M) &= FV(M) \setminus \{x\} \\
  \\
  BV(x) &= \emptyset \\
  BV(M N) &= BV(M) \cup BV(N) \\
  BV(\lambda x. M) &= \{x\} \cup BV(M)
\end{align*}

Применение абстракции к некоторому аргументу $(\lambda x.M) N$ -- это \textit{подстановка} $M[x \mapsto N]$ терма $N$ вместо \textit{свободных} вхождений переменной $x$ в терме $M$. Формально, правила подстановки:
\begin{align*}
  x[x \mapsto N] &= N \\
  y[x \mapsto N] &= y, (x \neq y) \\
  (T S)[x \mapsto N] &= T[x \mapsto N] S[x \mapsto N] \\
  (\lambda x.T)[x \mapsto N] &= \lambda x.T \\
  (\lambda y.T)[x \mapsto N] &= \lambda y.T[x \mapsto N], (y \notin FV(N), x \neq y)
\end{align*}

Рассмотрим, что произойдет, если в последнем правиле условие $ y \notin FV(N)$ не выполняется:
$$ (\lambda y.x)[x \mapsto y] = \lambda y.y $$

Получилось, что в результате подстановки мы превратили константную функцию $\lambda y.x$ в тождественную. Такая ситуация называется проблемой захвата переменной, когда при подстановке в $\lambda$-абстракцию переменные подставляемого терма захватываются абстракцией.

Эту проблему можно решить если принять так называемое соглашение Барендрегта о том, что имена связанных переменных всегда выбирать так, чтобы они отличались от имен свободных. В примере выше, например, мы можем переименовать связанную переменную $y$ в свежую $z$ и поведении абстракции $\lambda z.x$ не изменится. Тогда подстановку можно использовать без каких-либо оговорок о свободных и связанных переменных. Пример выше превратится в:
$$ (\lambda z.x)[x \mapsto y] = \lambda z.y $$

Как мы уже установили выше, мы можем переименовывать связанные переменные в абстракциях и их поведение при применении к аргументам не изменится. Более того, имена связанных переменных не играют для нас никакой роли. Поэтому, как правило, $\lambda$-термы и рассматривают с точностью до имен параметров абстракций. Формально, на множестве именованных термов $\Lambda$ можно задать отношение $\alphaeq \in \Lambda \times \Lambda$, которое называется $\alpha$-эквивалентностью и определяется как минимальное отношение конгруэнтности, порожденное следующими правилами:

\begin{center}
  \AxiomC{$x \in \mathcal{V}$}
  \UnaryInfC{$x \alphaeq x$}
  \DisplayProof{}
\end{center}

\begin{center}
  \AxiomC{$M \alphaeq M'$}
  \AxiomC{$N \alphaeq N'$}
  \BinaryInfC{$M N \alphaeq M' N'$}
  \DisplayProof{}
\end{center}

\begin{center}
  \AxiomC{$M[x \mapsto y] \alphaeq M'$}
  \AxiomC{$y \notin FV(M)$}
  \BinaryInfC{$\lambda x. M \alphaeq \lambda y.M'$}
  \DisplayProof{}
\end{center}

Наконец, сформулируем лемму о подстановке:

\begin{prop}
  \label{named:assoc}
  Для любых $T, M, N \in \Lambda; x, y \in \mathcal{V}$, если $x \neq y$ и $x \notin FV(M)$, то верно $T[x \mapsto N][y \mapsto M] = T[y \mapsto M][x \mapsto N[y \mapsto M]]$
\end{prop}

\begin{proof}
  Индукция по терму $T$. База индукции -- случай, когда $T$ является переменной. Рассмотрим три случая:
  \begin{enumerate}
    \item $T \equiv x$. Левая часть -- $x[x \mapsto N][y \mapsto M] = N[y \mapsto M]$. Правая часть -- $x[y\mapsto M][x \mapsto N[y \mapsto M]] = N[y \mapsto M]$.
    \item $T \equiv y$. Левая часть -- $y[x \mapsto N][y \mapsto M] = M$. Правая часть -- $y[y\mapsto M][x \mapsto N[y \mapsto M]] = M[x\mapsto N[y \mapsto M]]$. Так как $x \notin FV(M)$, то $M[x\mapsto N[y \mapsto M]] = M$
    \item $T \equiv z \neq x,y$. Обе части редуцируются к $z$.
  \end{enumerate}

  Случай аппликации тривиален, рассмотрим случай абстракции $\lambda z.T$. Рассмотрим возможные случаи:

  \begin{enumerate}
    \item $z \equiv x$. Левая часть -- $(\lambda x.T)[x \mapsto N][y \mapsto M] = (\lambda x.T)[y \mapsto M] \overset{\mathrm{x \notin FV(M)}}{=} \lambda x.T[y \mapsto M]$. Правая часть -- $(\lambda x.T)[y \mapsto M][x \mapsto N[y \mapsto M]] \overset{\mathrm{x \notin FV(M)}}{=} (\lambda x.T[y \mapsto M])[x \mapsto N[y \mapsto M]] = \lambda x.T[y \mapsto M]$.

    \item $z \equiv y$. Пусть $y \in FV(N)$ и $y \in FV(M)$. По соглашению Барендрегта, нам нужно переименовать $y$ в свежую переменную $y'$, такую что $y' \notin FV(N)$ и $y' \notin FV(M)$. Это эквивалентно тому, что мы можем осуществить подстановку $\lambda y'.t[y \mapsto y']$. Вычислим левую часть:
    \begin{gather*}
      (\lambda y'.T[y \mapsto y'])[x \mapsto N][y \mapsto M] = \\
       \lambda y'.T[y \mapsto y'][x \mapsto N][y \mapsto M] \overset{\mathrm{IH}}{=} \\
       \lambda y'.T[y \mapsto y'][y \mapsto M][x \mapsto N[y \mapsto M]]
    \end{gather*}

    Правая часть:
    \begin{gather*}
      (\lambda y'.T[y \mapsto y'])[y \mapsto M][x \mapsto N[y \mapsto M]] = \\
      \lambda y'.T[y \mapsto y'][y \mapsto M][x \mapsto N[y \mapsto M]]
    \end{gather*}

    Заметим здесь, что так как $y' \notin FV(N)$ и $y' \notin FV(M)$, то, очевидно, что $y' \notin FV(N[y \mapsto M])$, поэтому в последнем переходе обе подстановки <<проваливаются>> в абстракцию.

    \item $z \equiv y$. Пусть $y \in FV(N)$, но теперь уже $y \notin FV(M)$. Аналогично предыдущему пункту, мы переименовываем $y$ в свежую переменную $y'$, такую что $y' \notin FV(N)$ и $y' \notin FV(M)$, осуществляя подстановку $\lambda y'.t[y \mapsto y']$. Вычислим левую часть:
    \begin{gather*}
      (\lambda y'.T[y \mapsto y'])[x \mapsto N][y \mapsto M] = \\
       \lambda y'.T[y \mapsto y'][x \mapsto N][y \mapsto M] \overset{\mathrm{IH}}{=} \\
       \lambda y'.T[y \mapsto y'][y \mapsto M][x \mapsto N[y \mapsto M]]
    \end{gather*}

    А в этом случае, заметим, что так как мы сначала заменили \textbf{все} свободные вхождения $y$ в $T$ на $y'$, а потом на место $y$ подставили $M$, то вторая подстановка ничего не делает, поэтому левая часть окончательно равна $\lambda y'.T[y \mapsto y'][x \mapsto N[y \mapsto M]]$

    Правая часть тогда редуцируется в:
    \begin{gather*}
      (\lambda y.T)[y \mapsto M][x \mapsto N[y \mapsto M]] = \\
      (\lambda y.T)[x \mapsto N[y \mapsto M]] = \\
      \lambda y.T[x \mapsto N[y \mapsto M]]
    \end{gather*}

    В последнем шаге вычисления, отметим, что так как $y \in FV(N)$ и $y \notin FV(M)$, то $y \notin FV(N[y \mapsto M])$, поэтому подстановка снова заносится под абстракцию. Переименуем и здесь $y$ в $y'$, тогда получим:
    \begin{gather*}
      \lambda y.T[x \mapsto N[y \mapsto M]] \alphaeq \\
      \lambda y'.T[x \mapsto N[y \mapsto M]][y \mapsto y'] \overset{\mathrm{IH}}{=} \\
      \lambda y'.T[y \mapsto y'][x \mapsto N[y \mapsto M][y \mapsto y']]
    \end{gather*}

    Из тех же соображений, что и выше, подстановка $N[y \mapsto M][y \mapsto y']$ -- это то же самое, что и $N[y \mapsto M]$, поэтому правая часть окончательно вычисляется в $\lambda y'.T[y \mapsto y'][x \mapsto N[y \mapsto M]]$.

    \item $z \neq x,y$. Доказательство аналогично предыдущим двум пунктам. \qedhere

  \end{enumerate}
\end{proof}
