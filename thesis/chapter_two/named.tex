\subsection{Именованное представление термов}
\label{sec:named}

Тип данных для термов в именованном представлении выглядит следующим образом:

\begin{figure}[htp]
\centering
\begin{cminted}{haskell}
data Term =
  Var Name |
  App Term Term |
  Lam Name Term
\end{cminted}
\end{figure}

Здесь \mintinline{haskell}{Name} -- это множество переменных $\mathcal{V}$ из формального определения множества $\lambda$-термов $\Lambda$ в разделе~\ref{sec:lambda}.

Определим для такого представления операцию подстановки \mintinline{haskell}{subst}:

\begin{figure}[H]
\centering
\begin{cminted}{haskell}

subst :: Term -> Name -> Term -> Term
subst (Var x) x s = s
subst (Var y) x s = Var y
subst (App t1 t2) x s = App (subst t1 x s) (subst t2 x s)
subst (Lam x t) x s = Lam x t
subst (Lam y t) x s = subst (Lam x' (subst t y x')) x s

\end{cminted}
\end{figure}

В последнем определении подразумевается, что \mintinline{haskell}{x'} -- свежая переменная для терма \mintinline{haskell}{t}, которую мы подставляем, чтобы избежать захвата переменной.

Аналогично определим отношение $\alpha$-эквивалентности:
\begin{figure}[H]
\centering
\begin{cminted}{haskell}

-- TODO
data AlphaEq (t1 : Term) (t2 : Term)
  | AlphaEq (Var x) (Var y) => x = y
  | AlphaEq (App t1 s1) (App t2 s2) => AlphaEq t1 t2 && AlphaEq s1 s2
  | AlphaEq (Lam x1 t1) (Lam x2 t2) => AlphaEq t2 (subst t1 x1 (Var x2))

\end{cminted}
\end{figure}

Свойства подстановки выразятся тогда следующим образом:
\begin{figure}[H]
\centering
\begin{minted}{haskell}

subst-left-unit :: (t : Term) -> (x : Name) -> subst (Var x) x t = t
subst-right-unit :: (t : Term) -> (x : Name) -> subst t x (Var x) = t
subst-assoc :: (t M N : Term) -> (x y : Name) -> (Not (IsFree x M))
  -> subst (subst t x N) y M = subst (subst t y M) x (subst N y M)

\end{minted}
\end{figure}
