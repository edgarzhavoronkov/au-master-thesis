\subsection{Монадическое представление термов}
\label{sec:monad}
Существует еще один способ записи $\lambda$-термов, описанный в \cite{bird1999bruijn}. Библиотека \textbf{Bound}~\cite{bound} для языка \textbf{Haskell}, например, использует именно монадическое представление.

Прежде чем продолжить, мы дадим определение функтора и монады. Мы постараемся не перегружать читателя громоздкими определениями из теории категорий и попробуем дать неформальное, но интуитивно понятное определение этим двум объектам.

Чаще всего, имея на руках значение $v$ некоторого типа $V$ мы можем применить к нему функцию $f : V \to W$ и получить значение $f(v) \in W$. Теперь представим, что на руках у нас не просто значение, но значение в некотором контексте. О контексте, в свою очередь, можно думать как о контейнере, в котором лежит значение(например, как о списке значений или о дереве, в узлах которого содержатся значения\footnote{Пример с контейнером, в общем случае, не самый удачный, но он выбран по причине простоты. Примеры более сложных, вычислительных контекстов, таких как ввод-вывод или работа с состоянием в функциональном стиле не сильно коррелируют с определением лямбда-термов, поэтому специально исключены из рассмотрения.}). Теперь так просто применить функцию к значению в контексте мы уже не можем, нам нужно научиться каким-то образом извлекать значение из контекста, применять к нему функцию и запаковывать обратно.

Функтор делает именно это. Имея на входе функцию, которую мы хотим применить к значению и упакованное в контекст значение он извлекает значение из контекста, применяет к нему функцию и возвращает нам запакованное значение. В языке программирования \textbf{Haskell}, в классе типов \textit{Functor} определена операция \texttt{fmap}, которая и осуществляет все вышеперечисленное. Эта операция должна удовлетворять двум свойствам, которые еще называют <<законами функтора>>:

\begin{enumerate}
  \item Пусть $fv$ -- некоторое значение $v \in V$ в контексте. Тогда $\texttt{fmap}(id_{V}, fv) = fv$, где $id_{V}$ -- тождественное отображение на $V$.
  \item Пусть $fv$ -- некоторое значение $v \in V$ в контексте. Пусть $f : V \to W$, $g : W \to U$, тогда $\texttt{fmap}(g \circ f, fv) = \texttt{fmap}(g, \texttt{fmap}(f, fv))$
\end{enumerate}

Эти два закона говорят нам о том, что операция \texttt{fmap} не должна менять <<структуру>> контейнера. Например, она не должна порождать новых элементов в списке или переподвешивать поддеревья в дереве.

Заметим, что функция, которую применяют к упакованному значению функторы, принимает на вход и возвращает неупакованное значение. Монады же, в отличии от функторов, применяют к упакованному значению функцию, которая принимает \textbf{неупакованное} значение, а возвращает \textbf{упакованное} значение. Так же как и функтор, монада возвращает значение, упакованное в контекст.

Аналогично, в языке программирования \textbf{Haskell}, в классе типов \textit{Monad} определены две операции -- \texttt{return} и \texttt{>>=}(читается как <<bind>>, мы будем обозначать её соответственно \texttt{bind}). Операция \texttt{return} называется монадической единицей и принимает на вход значение $v \in V$ и возвращает упакованное в контекст значение. Операция \texttt{bind} называется монадическим связыванием и принимает на вход значение $mA$ типа $A$ в контексте $m$, функцию $k : A \to mB$ и возвращает значение $mB$ типа $B$ в контексте $m$. Функцию $k : A \to mB$ называют еще стрелкой Клейсли~\cite{kleisliArrows}. Эти две операции должны удовлетворять трем свойствам, которые носят название <<монадных законов>>:

\begin{enumerate}
  \item Пусть $m$ -- некоторое значение в контексте. Тогда $\texttt{bind}(m, \texttt{return}) = m$.
  \item Пусть $a$ -- некоторое неупакованное значение типа $A$ , а $k : A \to mB$. Тогда $\texttt{bind}(\texttt{return}(a), k) = k(a)$
  \item Пусть $m$ -- некоторое запакованное значение. Пусть $f : A \to mB$, $g : B \to mC$. Тогда $\texttt{bind}(\texttt{bind}(m, f), g) = \texttt{bind}(m, (x \mapsto \texttt{bind}(f(x), g)))$. Здесь и далее $x \mapsto f(x)$ обозначает анонимную функцию, сопоставляющую конкретному значению $x$ значение $f(x)$.
\end{enumerate}


Основная идея в том, что именованное представление для термов можно обобщить и свободные переменные брать из произвольного множества $V$. Тогда множество термов $\Lambda_{V}$ определяется индуктивно по следующим правилам:
\begin{center}
  \AxiomC{$v \in V$}
  \UnaryInfC{$v \in \Lambda_{V}$}
  \DisplayProof{}
\end{center}

\begin{center}
  \AxiomC{$M \in \Lambda_{V}$}
  \AxiomC{$N \in \Lambda_{V}$}
  \BinaryInfC{$M N \in \Lambda_{V}$}
  \DisplayProof{}
\end{center}

\begin{center}
  \AxiomC{$M \in \Lambda_{V \coprod \{*\}}$}
  \UnaryInfC{$\lambda M \in \Lambda_{V}$}
  \DisplayProof{}
\end{center}

Здесь $\{*\}$ -- это произвольное одноэлементное множество, а $\coprod$ -- операция размеченного объединения множеств. По определению, $A \coprod B$ состоит из элементов $inl(a)$ и $inr(b)$, где $a \in A$ и $b \in B$.  Так как для абстракции нам нужно иметь на одну свободную переменную больше, то ее можно получить взяв размеченное объединение с произвольным одноэлементным множеством. Это представление так же удобно для компьютерной реализации за счет того, что проверку корректности построения термов можно выполнять на уровне типов.

Пусть у нас есть терм $T \in \Lambda_{V}$ и функция $f : V \to W$, тогда мы можем задать функцию $\texttt{fmap} : \Lambda_{V} \to (V \to W) \to \Lambda_{W}$ рекурсивно:

\begin{enumerate}
  \item $v \mapsto f(v)$
  \item $M\ N \mapsto \texttt{fmap}(f, M)\ \texttt{fmap}(f, N)$
  \item $\lambda M \mapsto \lambda \texttt{fmap}(f'(f), M)$. Заметим, что просто так отобразить терм $M$ с помощью функции $f$ мы не можем, так как ее домен не совпадает со множеством, которым параметризован тип терма $M$. Поэтому мы построим по $f$ функцию $f'(f) : V \coprod \{*\} \to W \coprod \{*\}$. Устроена она будет следующим образом:
  \begin{enumerate}
    \item $f'(f)(inl(x)) = inl(f(x))$
    \item $f'(f)(inr(*)) = inr(*)$
  \end{enumerate}
\end{enumerate}

Интуитивно, действие опeрации \texttt{fmap} на терм -- это переименование переменных в нем. Покажем, что это определение операции \texttt{fmap} уважает тождественное отображение и композицию отображений.

\begin{prop}
  \label{monad:fmap-resp-id}
  Для любого $T \in \Lambda_{V}$ верно, что $\texttt{fmap}(id_{V}, T) = T$
\end{prop}

\begin{proof}
  Индукция по терму $T$. База тривиальна, равно как и случай аппликации, покажем, что утверждение верно и для случая лямбды. Вспомогательная функция $f'(f)$ устроена следующим образом:
  \begin{enumerate}
    \item $f'(id_{V})(inl(x)) = inl(x)$
    \item $f'(id_{V})(inr(*)) = inr(x)$
  \end{enumerate}
  Следовательно, оно является тождеством на $V \coprod \{*\}$. По индукционной гипотезе получаем, что случай для лямбды тоже верен.

  Формальное доказательство этого утверждения можно увидеть в приложении~\ref{apendix:monad}, в функции \texttt{fmap-respects-id}
\end{proof}

\begin{prop}
  \label{monad:fmap-resp-comp}
  Для любого $T \in \Lambda_{V}$ и $f : V \to W$, $g : W \to X$ верно, что $\texttt{fmap}(g \circ f, T) = \texttt{fmap}(f, \texttt{fmap}(g, T))$
\end{prop}

\begin{proof}
  Снова индукция по терму $T$. Случай переменной снова тривиален, случай аппликации напрямую следует из индукционной гипотезы, рассмотрим случай абстракции. Посмотрим, во что вычислится левая часть:
  \begin{gather*}
    \texttt{fmap}(g \circ f, \lambda M) = \lambda \texttt{fmap}(f'(g \circ f), M) \\
    \text{где} \\
    f'(g \circ f)(inl(v)) = inl(g(f(v))) \\
    f'(g \circ f)(inr(*)) = inr(*)
  \end{gather*}

  Правая, в свою очередь:
  \begin{gather*}
    \texttt{fmap}(g, \texttt{fmap}(f, \lambda M)) = \texttt{fmap}(g, \lambda \texttt{fmap}(f'(f), M)) = \\
    \lambda \texttt{fmap}(g'(g), \texttt{fmap}(f'(f), M))
  \end{gather*}

  Вспомогательная функция $f'(f) : V \coprod \{*\} \to W \coprod \{*\}$ устроена здесь следующим образом:
  \begin{enumerate}
    \item $f'(f)(inl(v)) = inl(f(v))$
    \item $f'(f)(inr(*)) = inr(*)$
  \end{enumerate}

  Вспомогательная функция $g'(g) : W \coprod \{*\} \to X \coprod \{*\}$ устроена здесь следующим образом:
  \begin{enumerate}
    \item $g'(g)(inl(w)) = g'(g)(inl(f(v))) = inl(g(f(v)))$
    \item $g'(g)(inr(*)) = inr(*)$
  \end{enumerate}

  Несложно увидеть, что эти две функции из левой и правой частей ведут себя одинаково, следовательно и для случая лямбды утверждение верно.
  Формальное доказательство этого утверждения можно увидеть в приложении~\ref{apendix:monad}, в функции \texttt{fmap-respects-сomp}
\end{proof}


Зададим структуру монады. Монадическая единица \texttt{return} устроена очень просто -- это переменная. Связывание \texttt{bind} определяется рекурсией по структуре терма $T$:

\begin{enumerate}
  \item $v \mapsto k(v)$
  \item $M\ N \mapsto (\texttt{bind}(M, k))\ (\texttt{bind}(N, k))$
  \item $\lambda M \mapsto \lambda(\texttt{bind}(M, k'(k)))$, где $k'(k) : V \coprod \{*\} \to \Lambda_{W \coprod \{*\}}$ и определяется следующим образом:
    \begin{enumerate}
      \item $k'(k)(inl(v)) = \texttt{fmap}(inl, k(v))$
      \item $k'(k)(inr(*)) = \texttt{return}(inr(*))$
    \end{enumerate}
\end{enumerate}

Действие монадического связывания можно, в свою очередь, проинтерпретировать как подстановку. Это утверждение не столь очевидно, но если рассмотреть сигнатуру \texttt{bind} и обратить внимание на то, что функцию $k : V \to \Lambda_{W}$ можно задать в виде списка пар $(V, \Lambda_{W})$, то это соответствие становится куда более явным. Монадные законы, в свою очередь, в точности описывают свойства подстановки, которые мы в явном виде задали в прошлых разделах~\ref{sec:named} и \ref{sec:index} -- унитальность и ассоциативность.

Сформулируем и докажем теперь свойства этих двух операций.

\begin{prop}
  \label{monad:bind-right-unit}
  Для любого $T \in \Lambda_{V}$ верно $\texttt{bind}(T, \texttt{return}) = T$.
\end{prop}

\begin{proof}
  Индукция по структуре терма $T$:
  \begin{enumerate}
    \item База индукции, $T = v$. Имеем, что $\texttt{bind}(v, \texttt{return}) = \texttt{return}(v) = v$.
    \item Случай аппликации напрямую следует из предположения индукции.
    \item Пусть теперь $T = \lambda M$. Имеем, что $\texttt{bind}(\lambda M, \texttt{return}) = \lambda (\texttt{bind}(M, k'(\texttt{return})))$. Вспомогательное отображение $k'(return)$ устроено следующим образом:
    \begin{enumerate}
      \item $k'(\texttt{return})(inl(v)) = \texttt{fmap}(inl, \texttt{return}(v)) = \texttt{return}(inl(v))$
      \item $k'(\texttt{return})(inr(*)) = \texttt{return}(inr(*))$
    \end{enumerate}
  \end{enumerate}

  Заметим, что оно ведет себя так же, как и \texttt{return} на $V \coprod \{*\}$, следовательно по индукционной гипотезе получаем исходное утверждение.
  Формальное доказательство этого утверждения можно увидеть в приложении~\ref{apendix:monad}, в функции \texttt{bind-right-unit}.
\end{proof}

\begin{prop}
  \label{monad:bind-left-unit}
  Для любого $v \in V$ и $k : V \to \Lambda_{V}$ верно $\texttt{bind}(\texttt{return}(v), k) = k(v)$
\end{prop}

\begin{proof}
  Утверждение тривиально следует из определения \texttt{bind} и того факта, что \texttt{return} -- это переменная. Формальное доказательство этого утверждения можно увидеть в приложении~\ref{apendix:monad}, в функции \texttt{bind-left-unit}.
\end{proof}

Прежде, чем формулировать последний монадный закон, сформулируем и докажем несколько технических лемм, которые помогут нам в его доказательстве.

\begin{lemma}
  \label{monad:bind-fmap-comm-lhs}
  Для любых $T \in \Lambda_{V}$, $f : V \to W$ и $k : W \to \Lambda_{U}$ верно $\texttt{bind}(\texttt{fmap}(f, T), k) = \texttt{bind}(T, k \circ f)$.
\end{lemma}

\begin{proof}
  Индукция по структуре терма $T$ с тривиальными случаями переменной и аппликации. Рассмотрим случай абстракции $\lambda M$. Нам нужно показать, что $\lambda \texttt{bind}(\texttt{fmap}(f'(f), M), k'(k)) = \lambda \texttt{bind}(M, k'(k \circ f))$. По индукционной гипотезе мы знаем, что $\texttt{bind}(\texttt{fmap}(f'(f), M), k'(k)) = \texttt{bind}(M, k'(k) \circ f'(f))$, Заметим теперь, что $k'(k \circ f)$ и $k'(k) \circ f'(f)$ ведут себя одинаково на всех входах, следовательно это утверждение доказано. Формальное доказательство этого утверждения можно увидеть в приложении~\ref{apendix:monad}, в функции \texttt{bind-fmap-comm-lhs}.
\end{proof}

\begin{lemma}
  \label{monad:bind-fmap-comm-rhs}
  Для любых $T \in \Lambda_{V}$, $f : W \to U$ и $k : V \to \Lambda_{W}$ верно $ \texttt{fmap}(f, \texttt{bind}(T, k)) = \texttt{bind}(T, (x \mapsto \texttt{fmap}(f, k(x))))$.
\end{lemma}

\begin{proof}
  Снова индукция по структуре терма $T$. Случай переменной и аппликации снова тривиален, поэтому рассмотрим случай абстракции $\lambda M$.

  Нам нужно показать, что $\lambda \texttt{fmap}(f'(f), \texttt{bind}(M, k'(k))) = \lambda \texttt{bind}(M, k'(x \mapsto \texttt{fmap}(f, k(x))))$. По индукционной гипотезе мы знаем, что $\texttt{fmap}(f'(f), \texttt{bind}(M, k'(k))) = \texttt{bind}(M, (x \mapsto \texttt{fmap}(f'(f), k'(k)(x))))$. Покажем теперь, что стрелки Клейсли $ k'(x \mapsto \texttt{fmap}(f, k(x))) $ и $ x \mapsto \texttt{fmap}(f'(f), k'(k)(x)) $ ведут себя одинаково на всех входах:

  \begin{enumerate}
    \item $inr(*)$. Обе стрелки вычисляются в $\texttt{return}(inr(*))$
    \item $inl(v)$. Надо показать, что $\texttt{fmap}(f'(inl), \texttt{fmap}(f, k(v))) = \texttt{fmap}(f'(f), \texttt{fmap}(inl, k(v)))$. Так как $\Lambda_{V}$ -- функтор и он уважает композицию отображений имеем, что нужно показать $\texttt{fmap}(f'(inl) \circ f, k(v)) = \texttt{fmap}(f'(f) \circ inl, k(v))$. Для этого в свою очередь нужно снова показать, что два отображения $f'(inl) \circ f$ и $f'(f) \circ inl$ ведут себя одинаково на всех входах, но это очень легко понять, просто взглянув на определение $f'$.
  \end{enumerate}

  Формальное доказательство этого утверждения можно увидеть в приложении~\ref{apendix:monad}, в функции \texttt{bind-fmap-comm-rhs}.
\end{proof}

\begin{lemma}
  \label{monad:bind-fmap-comm}
  Для любого $T \in \Lambda_{V}$ и $f : V \to \Lambda_{W}$ верно $\texttt{bind}(\texttt{fmap}(inl, t), k'(g)) = \texttt{fmap}(inl, \texttt{bind}(T, f))$.
\end{lemma}

\begin{proof}
  Снова индукция по структуре терма $T$, случай переменной и аппликации тривиален, рассмотрим случай абстракции $\lambda M$.

  Левая часть вычислится в:
  $$ \lambda \texttt{fmap}(f'(inl), \texttt{bind}(M, k'(f))) $$
  Правая в:
  $$ \lambda \texttt{bind}(\texttt{fmap}(f'(inl), M), k'(k'(f))) $$

  По лемме~\ref{monad:bind-fmap-comm-lhs} имеем, что $\texttt{bind}(\texttt{fmap}(f'(inl), M), k'(k'(f))) = \texttt{bind}(M, k'(k'(f)) \circ f'(inl))$. По лемме~\ref{monad:bind-fmap-comm-rhs} имеем $\texttt{fmap}(f'(inl), \texttt{bind}(M, k'(f))) = \texttt{bind}(M, (x \mapsto \texttt{fmap}(f'(inl), k'(f)(x))))$. Заметим теперь, что стрелки Клейсли $k'(k'(f)) \circ f'(inl)$ и $x \mapsto \texttt{fmap}(f'(inl), k'(f)(x))$ ведут себя одинаково на всех входах, тогда по симметричности и транзитивности равенства получаем доказательство требуемого утверждения.

  Формальное доказательство этого утверждения можно увидеть в приложении~\ref{apendix:monad}, в функции \texttt{bind-fmap-comm}.
\end{proof}

\begin{prop}
  \label{monad:bind-assoc}
  Для любого $T \in \Lambda_{V}$, $f : V \to \Lambda_{W}, g : W \to \Lambda_{U}$ верно $\texttt{bind}(\texttt{bind}(T, f), g) = \texttt{bind}(T, (x \mapsto \texttt{bind}(f(x), g)) )$.
\end{prop}

\begin{proof}
  Индукция по терму $T$:
  \begin{enumerate}
    \item Рассмотрим случай, когда $T = v$. Тогда левая часть вычисляется в $\texttt{bind}(f(v), g)$, ровно как и правая.
    \item Случай для аппликации следует напрямую из индукционной гипотезы.
    \item Рассмотрим случай абстракции $\lambda M$. Посмотрим, во что вычисляется левая часть:
    \begin{gather*}
      \texttt{bind}(\texttt{bind}(\lambda M, f), g) = \texttt{bind}(\lambda \texttt{bind}(M, k'(f)), g) = \\
      \lambda \texttt{bind}(\texttt{bind}(M, k'(f)), k'(g)) \\
      \text{где} \\
      k'(g)(inl(w)) = \texttt{fmap}(inl, g(w)) \\
      k'(g)(inr(*)) = \texttt{return}(inr(*)) \\
      \text{и} \\
      k'(f)(inl(v)) = \texttt{fmap}(inl, f(v)) \\
      k'(f)(inr(*)) = \texttt{return}(inr(*))
    \end{gather*}

    Правая часть, в свою очередь, вычисляется в:
    \begin{gather*}
      \texttt{bind}(\lambda M, (x \mapsto \texttt{bind}(f(x), g))) = \lambda \texttt{bind}(M, k'(x \mapsto \texttt{bind}(f(x), g)))
    \end{gather*}

    Чтобы воспользоваться индукционной гипотезой, необходимо показать, что $k'(x \mapsto \texttt{bind}(f(x), g)) : V \coprod \{*\} \to \Lambda_{U \coprod \{*\}}$ ведет себя так же как и $x \mapsto \texttt{bind}(k'(f)(x), k'(g))$. Для этого мы просто покажем, что они возвращают одинаковый результат на всех входах.

    Рассмотрим два случая, как могут выглядеть входные данные:
    \begin{enumerate}
      \item $inr(*)$. Обе части вычисляются в $\texttt{return}(inr(*))$.
      \item $inl(v)$. Левая часть вычисляется в: $$ \texttt{fmap}(inl, \texttt{bind}(f(v), g)) $$
      Правая: $$\texttt{bind}(\texttt{fmap}(inl, f(v)), f'(g))$$
      Воспользовавшись леммой~\ref{monad:bind-fmap-comm} для терма $f(v)$ и $g$ получаем доказательство исходного утверждения. \qedhere
    \end{enumerate}
  \end{enumerate}

  Формальное доказательство этого утверждения можно увидеть в приложении~\ref{apendix:monad}, в функции \texttt{bind-assoc}.
\end{proof}
