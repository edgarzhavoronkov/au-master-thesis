\subsection{Неименованное представление термов}

Как уже мы уже видели в предыдущем разделе, имена формальных параметров $\lambda$-абстракций не важны и, в целом, мы можем не обращать на них внимания. Более того, мы можем вообще отказаться от именованных переменных! Широко известен альтернативный способ записи термов через так называемые индексы Де Брауна(De Bruijn) -- \cite{nikolas1972bruijn}. В нем вместо имен переменных используются числовые индексы, показывающие сколько лямбд назад была захвачена переменная. Например комбинатор $S = \lambda f g x. f x (g x)$, записанный в таком представлении будет иметь следующий вид: $\lambda(\lambda(\lambda 3 1 (2 1)))$.

Существует и альтернативный способ такого представления. Множество всех термов разбивается на так называемые <<уровни>>(levels) и вместо него рассматриваются множества $\Lambda_{n}$, где $n$ -- длина контекста, в котором определен терм. О контексте в котором определен терм, можно думать, как о простом списке свободных переменных терма. Индуктивно, они определяются следующим образом:

\begin{center}
  \AxiomC{$0 \leqslant i < n$}
  \UnaryInfC{$v_{n, i} \in \Lambda_{n}$}
  \DisplayProof{}
\end{center}

\begin{center}
  \AxiomC{$T_{1} \in \Lambda_{n}$}
  \AxiomC{$T_{2} \in \Lambda_{n}$}
  \BinaryInfC{$T_{1} T_{2} \in \Lambda_{n}$}
  \DisplayProof{}
\end{center}

\begin{center}
  \AxiomC{$T \in \Lambda_{n + 1}$}
  \UnaryInfC{$\lambda T \in \Lambda_{n}$}
  \DisplayProof{}
\end{center}

В случае переменной индекс $i$ обозначает позицию переменной в контексте. Договоримся отсчитывать ее с конца контекста. Комбинатор $S$, например, в таком представлении будет выглядеть вот так: $\lambda (\lambda (\lambda v_{3,2} v_{3, 0} (v_{3, 1} v_{3, 0})))$.

Такое представление термов удобно потому что $\alpha$-эквивалентность сводится к самому обычному равенству и, как следствие, пропадает проблема коллизии имен переменных.

Определим операцию подстановки для таких термов. Мы определим ее в более общем случае -- вместо какой-то одной переменной мы будем осуществлять подстановку во \textbf{все} переменные терма. Пусть $T \in \Lambda_{n}$, и $S_{0}, \dots S_{n-1} \in \Lambda_{k}$. Тогда $subst(T, S_{n - 1}, \dots, S_{0}) \in \Lambda_{k}$ определяется следующим образом:
\begin{gather*}
  subst(v_{n, i}, S_{n - 1}, \dots, S_{0}) = S_{i} \\
  subst(T, v_{n, n-1}, \dots, v_{n, 0}) = T \\
  subst(T_{1} T_{2}, S_{n - 1}, \dots, S_{0}) = \\
  subst(T_{1}, S_{n - 1}, \dots, S_{0})\ subst(T_{2}, S_{n - 1}, \dots, S_{0}) \\
  subst(\lambda T, S_{n - 1}, \dots, S_{0}) = \lambda (subst(T, w(S_{n - 1}), \dots, w(S_{0}), v_{n+1, 0})
\end{gather*}

Операция $w(T)$ работает следующим образом. Пусть терм $T \in \Lambda_{n}$, тогда терм $w(T) \in \Lambda_{n+1}$ и определен как:
\begin{gather*}
  w(v_{n, i}) = v_{n+1, i+1} \\
  w(T_{1} T_{2}) = w(T_1)\ w(T_2) \\
  w(\lambda T) = \lambda (w(T))
\end{gather*}

Сформулируем и докажем вспомогательную лемму, которая пригодится нам далее:
\begin{lemma}
  \label{index:weak_lemma}
  Пусть $T \in \Lambda_{n}$, а $S_{n-1}, \dots, S_{0} \in \Lambda_{m}$. Тогда $subst(w(T), w(S_{n-1}), \dots, w(S_{0}), v_{m+1, 0}) = w(subst(T, S_{n-1}, \dots S_{0}))$
\end{lemma}

\begin{proof}
  Индукция по структуре терма $T$.
  \begin{enumerate}
    \item База индукции -- $v_{n, i}$. Левая часть равна, по определению подстановки:
    $$ subst(v_{n+1, i+1}, w(S_{n-1}), \dots, w(S_{0}), v_{m+1, 0}) = w(S_{i})$$
    Правая часть:
    $$w(subst(v_{n,i}, S_{n-1}, \dots, S_{0})) = w(S_{i})$$

    \item Случай аппликации снова тривиален. Рассмотрим случай абстракции $\lambda T$. Левая часть вычислится в:
    \begin{gather*}
        subst(w(\lambda T), w(S_{n-1}), \dots, w(S_{0}), v_{m+1, 0}) = \\
        subst(\lambda w(T), w(S_{n-1}), \dots, w(S_{0}), v_{m+1, 0}) = \\
        \lambda subst(w(T), w(w(S_{n-1})), \dots, w(w(S_{0})), v_{m+2, 1}, v_{m+2,0}) \overset{\mathrm{IH}}{=} \\
        \lambda w(subst(T, w(S_{n-1}), \dots, w(S_{0}), v_{m+1, 0}))
    \end{gather*}

    Правая часть вычисляется в:
    \begin{gather*}
        w(subst(\lambda T, S_{n-1}, \dots, S_{0})) = \\
        w(\lambda subst(T, w(S_{n-1}), \dots, w(S_{0}), v_{m+1, 0})) =
        \lambda w(subst(T, w(S_{n-1}), \dots, w(S_{0}), v_{m+1, 0}))
    \end{gather*}
  \end{enumerate}
\end{proof}

Аналогично именованному представлению, сформулируем лемму о подстановке:

\begin{prop}
  \label{index:assoc}
  Пусть $T \in \Lambda_{n}; T_{n - 1}, \dots T_{0} \in \Lambda_{m}, S_{m-1}, \dots S_{0} \in \Lambda_{k}$, тогда верно $subst(subst(T, T_{n - 1}, \dots T_{0}), S_{m-1}, \dots S_{0}) = subst(T, T_{n - 1}' \dots, T_{0}')$, где $T_{i}' = subst(T_{i}, S_{m-1}, \dots S_{0})$.
\end{prop}

Заметим еще, что в таком представлении термов нет необходимости в сторонних условиях, как в лемме о подстановке для именованных термов.

\begin{proof}
  Это утверждение точно так же доказывается индукцией по структуре терма $T$. База индукции -- случай, когда терм представляет собой переменную $v_{n, i}$. Тогда левая часть вычисляется в $subst(T_{i}, S_{m-1}, \dots S_{0})$, ровно как и правая.

  Случай аппликации снова тривиален, рассмотрим случай абстракции $\lambda T$. Вычислим левую часть:
  \begin{gather*}
    subst(subst(\lambda T, T_{n-1}, \dots, T_{0}), S_{m-1}, \dots, S_{0}) = \\
    subst(\lambda subst( T, w(T_{n - 1}), \dots w(T_{0}), v_{m+1, 0} ), S_{m - 1}, \dots, S_{0}) = \\
    \lambda(subst(subst( T, w(T_{n - 1}), \dots w(T_{0}), v_{m+1, 0} ), w(S_{m-1}), \dots, w(S_{0}), v_{k+1, 0}) \overset{\mathrm{IH}}{=} \\
    \lambda(subst(T, subst(w(T_{n-1}), w(S_{m-1}), \dots, w(S_{0}), v_{k+1, 0}), \dots, \\
    subst(w(T_{0}), w(S_{m-1}), \dots, w(S_{0}), v_{k+1, 0}), subst(v_{m+1, 0}, w(S_{m-1}), \dots, w(S_{0}), v_{k+1, 0}))) = \\
    \lambda(subst(T, subst(w(T_{n-1}), w(S_{m-1}), \dots, w(S_{0}), v_{k+1, 0}), \dots, \\
    subst(w(T_{0}), w(S_{m-1}), \dots, w(S_{0}), v_{k+1, 0}), v_{k+1, 0}))
  \end{gather*}

  Вычислим теперь правую часть:
  \begin{gather*}
    subst(\lambda T, subst(T_{n-1}, S_{m - 1}, \dots, S_{0}), \dots, subst(T_{0}, S_{m - 1}, \dots, S_{0})) = \\
    \lambda(subst(T, w(subst(T_{n-1}, S_{m - 1}, \dots, S_{0})), \dots, w(subst(T_{0}, S_{m - 1}, \dots, S_{0})), v_{k+1, 0}))
  \end{gather*}

  По лемме~\ref{index:weak_lemma} $subst(w(T_{i}), w(S_{m-1}), \dots, w(S_{0}), v_{k+1, 0}) = w(subst(T_{i}, S_{m - 1}, \dots, S_{0}))$ для всех $i=\overline{0, n-1}$, следовательно наше утверждение верно.
\end{proof}
