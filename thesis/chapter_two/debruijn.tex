\subsection{Неименованное представление термов}

Как уже упоминалось в введении, именованное представление -- не единственный способ записи $\lambda$-термов. Помимо него широко известен способ записи через так называемые индексы Де Брауна(De Bruijn) -- \cite{nikolas1972bruijn}. В нем вместо имен переменных используются числовые индексы, показывающие сколько лямбд назад была захвачена переменная. Например комбинатор $S = \lambda f g x. f x (g x)$, записанный в таком представлении будет иметь следующий вид: $\lambda(\lambda(\lambda 3 1 (2 1)))$.

Существует и альтернативный способ такого представления. Множество всех термов разбивается на так называемые <<уровни>>(levels) и вместо него рассматриваются множества $\Lambda_{n}$, где $n$ -- длина контекста, в котором определен терм. Индуктивно, они определяются следующим образом:

\begin{center}
  \AxiomC{$0 \leqslant i < n$}
  \UnaryInfC{$v_{n, i} \in \Lambda_{n}$}
  \DisplayProof{}
\end{center}

\begin{center}
  \AxiomC{$t_{1} \in \Lambda_{n}$}
  \AxiomC{$t_{2} \in \Lambda_{n}$}
  \BinaryInfC{$t_{1} t_{2} \in \Lambda_{n}$}
  \DisplayProof{}
\end{center}

\begin{center}
  \AxiomC{$t \in \Lambda_{n + 1}$}
  \UnaryInfC{$\lambda t \in \Lambda_{n}$}
  \DisplayProof{}
\end{center}

Комбинатор $S$ в таком представлении будет выглядеть вот так: $\lambda (\lambda (\lambda v_{3,2} v_{3, 0} (v_{3, 1} v_{3, 0})))$.

Такое представление термов удобно потому что $\alpha$-эквивалентность сводится к самому обычному равенству и, как следствие, пропадает проблема коллизии имен переменных.
