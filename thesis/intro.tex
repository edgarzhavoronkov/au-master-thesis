\section*{Введение}

Верификация программного обеспечения используется в важнейших отраслях индустрии разработки ПО и включает в себя формальные рассуждения о программах и их свойствах. Так как языки программирования используют связывание имен, то в программах на них зачастую возникают проблема коллизий в именах переменных, функций и других конструкций.

Соответственно, для того, чтобы формально рассуждать о свойствах программ необходима формальная система, позволяющая решать вышеозначенную проблему. Пример такой системы -- это $\lambda$-исчисление, лежащее в основе функционального программирования. В работе делается попытка формализовать различные представления этой системы и понять, с каким из них наиболее удобно работать.

В первой главе будет дан анализ предметной области и краткое введение в $\lambda$-исчисление. Мы обозначим цель работы и задачи, решение которых необходимо для её достижения. Будут рассмотрены уже существующие решения и описаны их достоинства и недостатки.

Во второй главе будут описаны три представления $\lambda$-термов. Для каждого из представлений мы определим характерные свойства и покажем, почему они верны. Кроме того, мы покажем, что эти представления эквивалентны между собой.

В третьей главе предполагается описание деталей реализации. Мы опишем язык, с помощью которого предлагается формализовать рассмотренные во второй главе представления а так же тонкие моменты, с которыми пришлось столкнуться в ходе выполнения работы.
