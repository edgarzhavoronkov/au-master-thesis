\section*{Введение}

Система типов в языке программирования позволяет выражать свойства программы, написанной на этом языке. Соответственно, чем мощнее система типов, тем больше инвариантов программ мы можем выразить. Языки с зависимыми типами являют собой венец эволюции и позволяют в полной мере описывать спецификацию программ. Кроме того, являясь полноценной логикой, эти языки могут использоваться для формализации математики. В работе делается попытка формализовать некоторые свойства $\lambda$-исчисления -- формальной системы, лежащей в основе функционального программирования.

В первой главе будет дан анализ предметной области, краткое введение в $\lambda$-исчисление. Будет дана мотивация к постановке цели, непосредственно цель и задачи, решение которых необходимо для её достижения. Будут рассмотрены уже существующие решения и описаны их достоинства и недостатки.

Во второй главе будут рассмотрены различные представления $\lambda$-термов. Предполагается описание различных свойств, присущих каждому из представлений и их формализация.
