\section*{Введение}

Верификация программного обеспечения используется в важнейших отраслях индустрии разработки программного обеспечения и включает в себя формальные рассуждения о программах и их свойствах. Так как языки программирования используют имена для идентификации сущностей(данные, функции и другие конструкции), то в программах зачастую возникают проблема коллизий. Она заключается в том, что разные сущности идентифицируются одним именем.

Языки программирования или средства для разработки пытаются сами решить эту проблему. Например анализатор кода в среде разработки может подсказать о том, что такое имя уже занято и программисту следует придумать новое. Некоторые языки разрешают имена с помощью системы модулей(или пространств имен, как например в языке C++).

Более того, иногда возникает желание рассматривать конструкции в языке программирования с точностью до имен параметров. Это полезно, например, при поиске фрагментов кода, которые производят одинаковые вычисления или подчиняются некоторому общему шаблону. Пример таких фрагментов -- всевозможные циклы и обходы контейнеров.

Соответственно, для того, чтобы доказывать какие-либо свойства программ необходима формальная система, позволяющая решать проблемы коллизий имен и формально описывающая отношение так называемой $\alpha$-эквивалентности -- эквивалентности в поведении конструкций с разными именами формальных параметров. Пример такой системы -- это $\lambda$-исчисление, лежащее в основе функциональных языков программирования. С помощью фундаментальной операции подстановки, мы можем не только записывать лямбда-выражения\footnote{Далее мы будем использовать термины <<лямбда-терм>>, <<$\lambda$-терм>>, <<лямбда-выражение>>, полагая их синонимами.}, но и сокращать их, другими словами, мы можем осуществлять процесс вычисления. Естественно, что операция подстановки удовлетворяет некоторым базовым свойствам, которые мы опишем далее.

Существуют различные представления этой системы, по-разному решающие описанные выше проблемы. Например, в именованном представлении у переменных есть имена и нам нужно рассматривать выражения с точностью до имен параметров функций. Существует неименованное представление, в котором, например, переменная -- это индекс в некотором глобальном хранилище переменных. Эти идеи можно обобщить и рассуждать о выражениях, используя методы теории категорий.

В работе делается попытка установить равенство между различными представлениями этой системы. Для этого мы планируем реализовать с помощью системы автоматического доказательства теорем эти представления, построить соответствия между этими представлениями и доказать их взаимно-однозначность.

В первой главе будет дан анализ предметной области, краткое описание $\lambda$-исчисления. Мы обозначим цель работы и задачи, решение которых необходимо для её достижения. Кроме того, будут рассмотрены уже существующие решения и описаны их отличия от представленного в работе.

Во второй главе будут описаны три представления $\lambda$-термов. Для каждого из представлений мы определим характерные свойства и покажем, почему они верны. Кроме того, мы покажем, что эти представления эквивалентны между собой.

В третьей главе предполагается описание деталей реализации. Мы опишем язык, с помощью которого предлагается формализовать рассмотренные во второй главе представления а так же тонкие моменты, с которыми пришлось столкнуться в ходе выполнения работы.
