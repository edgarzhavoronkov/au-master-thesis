\section*{Введение}

\paragraph{Мотивация}
Верификация программного обеспечения используется в важнейших отраслях индустрии разработки программного обеспечения и включает в себя формальные рассуждения о программах и их свойствах. Так как языки программирования используют связывание имен, то в программах на них зачастую возникают проблема коллизий в именах переменных, функций и других конструкций.

Языки программирования или средства для разработки пытаются сами решить эту проблему. Например анализатор кода в среде разработке может подсказать о том, что такое имя уже занято и программисту следует придумать новое. Некоторые языки разрешают имена с помощью системы модулей(или пространств имен, как например в языке C++).

Более того, иногда возникает желание рассматривать конструкции в языке программирования с точностью до имен параметров. Это полезно, например, при поиске фрагментов кода, которые производят одинаковые вычисления или подчиняются некоторому общему шаблону. Пример таких фрагментов -- всевозможные циклы и обходы контейнеров.

Соответственно, для того, чтобы доказывать какие-либо свойства программ необходима формальная система, позволяющая решать проблему коллизий имен и формально описывающая отношение так называемой $\alpha$-эквивалентности -- эквивалентности в поведении конструкций с разными именами формальных параметров. Пример такой системы -- это $\lambda$-исчисление, лежащее в основе функционального программирования. В работе делается попытка формализовать различные представления этой системы и понять, с каким из них наиболее удобно работать.

\paragraph{План работы}

В первой главе будет дан анализ предметной области, краткое описание $\lambda$-исчисления. Мы обозначим цель работы и задачи, решение которых необходимо для её достижения. Кроме того, будут рассмотрены уже существующие решения и описаны их отличия от представленного в работе.

Во второй главе будут описаны три представления $\lambda$-термов. Для каждого из представлений мы определим характерные свойства и покажем, почему они верны. Кроме того, мы покажем, что эти представления эквивалентны между собой.

В третьей главе предполагается описание деталей реализации. Мы опишем язык, с помощью которого предлагается формализовать рассмотренные во второй главе представления а так же тонкие моменты, с которыми пришлось столкнуться в ходе выполнения работы.
