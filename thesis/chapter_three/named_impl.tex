\subsection{Реализация именованного представления термов}

Примерный план того, что здесь будет:
\begin{enumerate}
  \item Тип данных для термов. Сказать, что натуральные числа, так как строк в языке не завезли и на натуральных числах разрешимое равенство, это пригодится в дальнейшем
  \item Сказать про предикат, что терм определен в контексте, сказать, что мы тут пользуемся левой альфа-эквивалентностью, так как для нормальной надо делать кучу телодвижений. Про нормальную тоже сказать, что пришлось определять предикат "переменная свежа в терме". Если определять проще, то некоторые свойства не завершаются(например рефлексивность использует унитальность и проверке на завершаемость неочевидно, что они завершаются)
  \item Сказать, что подстановка определяется крайне неудобно, мы определили более общий случай -- параллельной подстановки. Причем сначала мы думали вкрутить туда функцию, которая переменные переименовывает, но столкнулись с техническими проблемами в самом языке(там нормализация не дорабатывала)
  \item ОЧЕНЬ АККУРАТНО сказать про то, что лемма о подстановке довольно нетривиальное утверждение для формального доказательства. Неформально, мы доказали ее, крайне небрежно используя индукционную гипотезу. Если бы мы писали такое доказательство в нашей системе, то было бы неочевидно, что оно завершается. Поэтому работу хотелось бы закончить на проблемах, с которыми пришлось столкнуться.
\end{enumerate}
