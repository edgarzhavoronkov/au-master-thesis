\section{Представления \texorpdfstring{$\lambda$}{лямбда}-термов}

В этой главе мы опишем три представления $\lambda$-термов: именованное, неименованное(через индексы Де Брауна) и монадическое. Мы формально опишем, как определяются термы в каждом из представлений, определим характерные свойства для каждого представления и покажем, почему они верны. Кроме того, мы установим соответствие между всеми представлениями.

\subsection{Именованное представление термов}
\label{sec:named}

Термы $\lambda$-исчисления($\lambda$-термы) в именованном представлении конструируются из переменных путем применения друг к другу или создания анонимных функций.

Формально, пусть $\mathcal{V}=\{x,y,z,\dots\}$ -- счетное множество переменных. Договоримся обозначать переменные прописными буквами, а произвольные термы -- заглавными. Тогда множество $\lambda$-термов $\Lambda$ определяется индуктивно, согласно следующим правилам:
\begin{align*}
  v \in \mathcal{V} &\Rightarrow v \in \Lambda \\
  M, N \in \Lambda &\Rightarrow M N \in \Lambda \\
  v \in \mathcal{V}, M \in \Lambda &\Rightarrow \lambda v.M \in \Lambda
\end{align*}

Нотация аппликации $M N$ обозначает применение функции $M$ ко входу $N$. Заметим, что так как здесь не вводится никаких правил типизации, то ничто не мешает нам применить терм к самому себе(i.e $F F$). Нотация абстракции $\lambda x.M$, в свою очередь, обозначает создание анонимной функции от переменной $x$, которая сопоставляет конкретному значению $x$ выражение $M[x]$. Здесь заметим, что терм $M$ вовсе не обязан содержать в себе переменную $x$, в таком случае мы считаем абстракцию $\lambda x.M$ константной функцией.

Некоторые примеры термов:
\begin{gather*}
   \lambda x.x \\
   \lambda x y.x \\
   (\lambda x.f (x x)) (\lambda x.f (x x))
\end{gather*}

Переменная $x$ после абстракции $\lambda x.M$ называется \textit{связанной}. Соответственно, до абстракции она была \textit{свободной}. Формально, множества $FV(T)$ свободных и $BV(T)$ связанных переменных терма $T$ определяются индуктивно следующим образом:
\begin{align*}
  FV(x) &= \{x\} \\
  FV(M N) &= FV(M) \cup FV(N) \\
  FV(\lambda x. M) &= FV(M) \setminus \{x\} \\
  \\
  BV(x) &= \emptyset \\
  BV(M N) &= BV(M) \cup BV(N) \\
  BV(\lambda x. M) &= \{x\} \cup BV(M)
\end{align*}

Применение абстракции к некоторому аргументу $(\lambda x.M) N$ -- это \textit{подстановка} $M[x \mapsto N]$ терма $N$ вместо \textit{свободных} вхождений переменной $x$ в терме $M$. Формально, правила подстановки:
\begin{align*}
  x[x \mapsto N] &= N \\
  y[x \mapsto N] &= y, (x \neq y) \\
  (T S)[x \mapsto N] &= T[x \mapsto N] S[x \mapsto N] \\
  (\lambda x.T)[x \mapsto N] &= \lambda x.T \\
  (\lambda y.T)[x \mapsto N] &= \lambda y.T[x \mapsto N], (y \notin FV(N), x \neq y)
\end{align*}

Рассмотрим, что произойдет, если в последнем правиле условие $ y \notin FV(N)$ не выполняется:
$$ (\lambda y.x)[x \mapsto y] = \lambda y.y $$

Получилось, что в результате подстановки мы превратили константную функцию $\lambda y.x$ в тождественную. Такая ситуация называется проблемой захвата переменной, когда при подстановке в $\lambda$-абстракцию переменные подставляемого терма захватываются абстракцией.

Эту проблему можно решить если принять так называемое соглашение Барендрегта о том, что имена связанных переменных всегда выбирать так, чтобы они отличались от имен свободных. В примере выше, например, мы можем переименовать связанную переменную $y$ в свежую $z$ и поведении абстракции $\lambda z.x$ не изменится. Тогда подстановку можно использовать без каких-либо оговорок о свободных и связанных переменных. Пример выше превратится в:
$$ (\lambda z.x)[x \mapsto y] = \lambda z.y $$

Как мы уже установили выше, мы можем переименовывать связанные переменные в абстракциях и их поведение при применении к аргументам не изменится. Более того, имена связанных переменных не играют для нас никакой роли. Поэтому, как правило, $\lambda$-термы и рассматривают с точностью до имен параметров абстракций. Формально, на множестве именованных термов $\Lambda$ можно задать отношение $\alphaeq \in \Lambda \times \Lambda$, которое называется $\alpha$-эквивалентностью и определяется как минимальное отношение конгруэнтности, порожденное следующими правилами:

\begin{center}
  \AxiomC{$x \in \mathcal{V}$}
  \UnaryInfC{$x \alphaeq x$}
  \DisplayProof{}
\end{center}

\begin{center}
  \AxiomC{$M \alphaeq M'$}
  \AxiomC{$N \alphaeq N'$}
  \BinaryInfC{$M N \alphaeq M' N'$}
  \DisplayProof{}
\end{center}

\begin{center}
  \AxiomC{$M[x \mapsto y] \alphaeq M'$}
  \AxiomC{$y \notin FV(M)$}
  \BinaryInfC{$\lambda x. M \alphaeq \lambda y.M'$}
  \DisplayProof{}
\end{center}

Наконец, сформулируем лемму о подстановке:

\begin{prop}
  \label{named:assoc}
  Для любых $T, M, N \in \Lambda; x, y \in \mathcal{V}$, если $x \neq y$ и $x \notin FV(M)$, то верно $T[x \mapsto N][y \mapsto M] = T[y \mapsto M][x \mapsto N[y \mapsto M]]$
\end{prop}

\begin{proof}
  Индукция по терму $T$. База индукции -- случай, когда $T$ является переменной. Рассмотрим три случая:
  \begin{enumerate}
    \item $T \equiv x$. Левая часть -- $x[x \mapsto N][y \mapsto M] = N[y \mapsto M]$. Правая часть -- $x[y\mapsto M][x \mapsto N[y \mapsto M]] = N[y \mapsto M]$.
    \item $T \equiv y$. Левая часть -- $y[x \mapsto N][y \mapsto M] = M$. Правая часть -- $y[y\mapsto M][x \mapsto N[y \mapsto M]] = M[x\mapsto N[y \mapsto M]]$. Так как $x \notin FV(M)$, то $M[x\mapsto N[y \mapsto M]] = M$
    \item $T \equiv z \neq x,y$. Обе части редуцируются к $z$.
  \end{enumerate}

  Случай аппликации тривиален, рассмотрим случай абстракции $\lambda z.T$. Пусть $z \neq x,y$ и $z \in FV(N)$. По соглашению Барендрегта, нам нужно переименовать $z$ в свежую переменную $z'$, такую что $z' \notin FV(N)$, $z' \notin FV(M)$ и $z \notin FV(N[y \mapsto M])$. Это эквивалентно тому, что мы можем осуществить подстановку $\lambda z'.t[z \mapsto z']$. Вычислим левую часть:
  \begin{gather*}
    (\lambda z'.t[z \mapsto z'])[x \mapsto N][y \mapsto M] = \\
    \lambda z'.t[z \mapsto z'][x \mapsto N][y \mapsto M] \overset{\mathrm{IH}}{=} \\
    \lambda z'.t[z \mapsto z'][y \mapsto M][x \mapsto N[y \mapsto M]]
  \end{gather*}

  Правая часть:
  \begin{gather*}
    (\lambda z'.t[z \mapsto z'])[y \mapsto M][x \mapsto N[y \mapsto M]] = \\
    \lambda z'.t[z \mapsto z'][y \mapsto M][x \mapsto N[y \mapsto M]]
  \end{gather*}

  Остальные случаи проверяются аналогично, мы не будем приводить их здесь в целях экономии места.
\end{proof}

\subsection{Неименованное представление термов}

Как уже упоминалось в введении, именованное представление -- не единственный способ записи $\lambda$-термов. Помимо него широко известен способ записи через так называемые индексы Де Брауна(De Bruijn) -- \cite{nikolas1972bruijn}. В нем вместо имен переменных используются числовые индексы, показывающие сколько лямбд назад была захвачена переменная. Например комбинатор $S = \lambda f g x. f x (g x)$, записанный в таком представлении будет иметь следующий вид: $\lambda(\lambda(\lambda 3 1 (2 1)))$.

Существует и альтернативный способ такого представления. Множество всех термов разбивается на так называемые <<уровни>>(levels) и вместо него рассматриваются множества $\Lambda_{n}$, где $n$ -- длина контекста, в котором определен терм. Индуктивно, они определяются следующим образом:

\begin{center}
  \AxiomC{$0 \leqslant i < n$}
  \UnaryInfC{$v_{n, i} \in \Lambda_{n}$}
  \DisplayProof{}
\end{center}

\begin{center}
  \AxiomC{$t_{1} \in \Lambda_{n}$}
  \AxiomC{$t_{2} \in \Lambda_{n}$}
  \BinaryInfC{$t_{1} t_{2} \in \Lambda_{n}$}
  \DisplayProof{}
\end{center}

\begin{center}
  \AxiomC{$t \in \Lambda_{n + 1}$}
  \UnaryInfC{$\lambda t \in \Lambda_{n}$}
  \DisplayProof{}
\end{center}

Комбинатор $S$ в таком представлении будет выглядеть вот так: $\lambda (\lambda (\lambda v_{3,2} v_{3, 0} (v_{3, 1} v_{3, 0})))$.

Такое представление термов удобно потому что $\alpha$-эквивалентность сводится к самому обычному равенству и, как следствие, пропадает проблема коллизии имен переменных.

\subsection{Монадическое представление термов}

Именованное представление для термов можно обобщить и свободные переменные брать из произвольного множества -- \cite{bird1999bruijn}

Легко заметить, что это функтор, доказательство этого факта тривиально.

Чуть менее очевидно, но это монада, это утверждение не так очевидно, поэтому мы его докажем.

\subsection{Преобразования между представлениями}

В этом разделе мы опишем преобразования между представлениями и начнем с преобразования именованных термов в неименованные. Очевидно, что для осуществления этого нам необходимо знать порядок на переменных в терме. Поэтому мы считаем, что кроме самого терма нам дают контекст -- список свободных переменных терма, не содержащий повторений. Тот факт, что терм $T$ определен в контексте $\Gamma$ традиционно будем обозначать $\Gamma \vdash T$.

Итак, преобразование $\Phi$ именованных термов в неименованные принимает на вход контекст $\Gamma = x_{1}, \dots, x_{n}$, терм $T \in \Lambda$  такой что все его свободные переменные присутствуют в $\Gamma$ и возвращает неименованный терм $T' \in \Lambda_{n}$. Определяется оно индукцией по структуре терма $T$:

\begin{enumerate}
  \item $x_{1}, \dots, x_{n} \vdash x_{i} \mapsto v_{n, n - i}$
  \item $x_{1}, \dots, x_{n} \vdash M N \mapsto \Phi(x_{1}, \dots, x_{n} \vdash M)\ \Phi(x_{1}, \dots, x_{n} \vdash N)$
  \item $x_{1}, \dots, x_{n} \vdash \lambda x.M \mapsto \lambda \Phi(x_{1}, \dots, x_{n} \cup \{x\} \vdash M)$
\end{enumerate}

Покажем, что такое преобразование уважает отношение $\alpha$-эквивалентности, введенное в разделе~\ref{sec:named}.

\begin{prop}
  Пусть $T_{1}, T_{2} \in \Lambda$, $\Gamma \vdash T_{1}$, $\Gamma \vdash T_{2}$ и $T_{1} \alphaeq T_{2}$. Тогда $\Phi(\Gamma \vdash T_{1}) = \Phi(\Gamma \vdash T_{2})$.
\end{prop}

\begin{proof}
  Индукция по термам $T_{1}$ и $T_{2}$. Так как мы знаем, что они $\alpha$-эквивалентны, то нам нет необходимости рассматривать всевозможные комбинации термов. Поэтому рассмотрим лишь случаи, когда термы имеют общую структуру(две переменные, две аппликации или две абстракции).

  База индукции для случая двух переменных тривиальна, как и случай для двух аппликаций. Рассмотрим случай, когда $T_{1} = \lambda x.M$, $T_{2} = \lambda y.M'$. Так как они $\alpha$-эквивалентны, то мы знаем, что $M' \alphaeq M[x \mapsto y]$ и $y \notin FV(M)$. Мы знаем, что $\Gamma \vdash \lambda x.M$ и $\Gamma \vdash \lambda y.M'$, следовательно $\Gamma \cup \{x\} \vdash M$ и $\Gamma \cup \{y\} \vdash M'$. Кроме того, так как $y \notin FV(M)$, то $\Gamma \cup \{y\} \vdash M[x \mapsto y]$. По индукционной гипотезе мы знаем, что $M' \alphaeq M[x \mapsto y]$, а так как они определены в одинаковых контекстах, то и $\Phi(\Gamma \cup \{y\} \vdash M') = \Phi(\Gamma \cup \{y\} \vdash M[x \mapsto y])$. Отсюда получаем требуемое.
\end{proof}

Легко сконструировать преобразование в обратную сторону -- из неименованных термов в именованные. Кроме того, легко показать, что они взаимно-обратны.

