\section{Представления \texorpdfstring{$\lambda$}{лямбда}-термов}

В этом разделе мы опишем три представления $\lambda$-термов: именованное, неименованное(через индексы Де Брауна) и монадическое. Мы опишем, как кодируются типы данных для термов в синтаксисе \textbf{Haskell}-подобного языка, определим характерные свойства для каждого представления и покажем, почему они верны. Кроме того, мы установим соответствие между всеми представлениями.

\subsection{Именованное представление термов}
\label{sec:named}

Термы $\lambda$-исчисления($\lambda$-термы) в именованном представлении конструируются из переменных путем применения друг к другу или создания анонимных функций.

Формально, пусть $\mathcal{V}=\{x,y,z,\dots\}$ -- счетное множество переменных. Договоримся обозначать переменные прописными буквами, а произвольные термы -- заглавными. Тогда множество $\lambda$-термов $\Lambda$ определяется индуктивно, согласно следующим правилам:
\begin{center}
  \AxiomC{$v \in \mathcal{V}$}
  \UnaryInfC{$v \in \Lambda$}
  \DisplayProof{}
\end{center}

\begin{center}
  \AxiomC{$M \in \Lambda$}
  \AxiomC{$N \in \Lambda$}
  \BinaryInfC{$M N \in \Lambda$}
  \DisplayProof{}
\end{center}

\begin{center}
  \AxiomC{$M \in \Lambda$}
  \AxiomC{$v \in \mathcal{V}$}
  \BinaryInfC{$\lambda v.M \in \Lambda$}
  \DisplayProof{}
\end{center}

Нотация аппликации $M N$ обозначает применение функции $M$ ко входу $N$. Заметим, что так как здесь не вводится никаких правил типизации, то ничто не мешает нам применить терм к самому себе(i.e $F F$). Нотация абстракции $\lambda x.M$, в свою очередь, обозначает создание анонимной функции от переменной $x$, которая сопоставляет конкретному значению $x$ выражение $M[x]$. Здесь заметим, что терм $M$ вовсе не обязан содержать в себе переменную $x$, в таком случае мы считаем абстракцию $\lambda x.M$ константной функцией.

Некоторые примеры термов:
\begin{gather*}
   \lambda x.x \\
   \lambda x y.x \\
   (\lambda x.f (x x)) (\lambda x.f (x x))
\end{gather*}

Переменная $x$ после абстракции $\lambda x.M$ называется \textit{связанной}. Соответственно, до абстракции она была \textit{свободной}. Формально, множества $FV(T)$ свободных и $BV(T)$ связанных переменных терма $T$ определяются индуктивно следующим образом:
\begin{align*}
  FV(x) &= \{x\} \\
  FV(M N) &= FV(M) \cup FV(N) \\
  FV(\lambda x. M) &= FV(M) \setminus \{x\} \\
  \\
  BV(x) &= \emptyset \\
  BV(M N) &= BV(M) \cup BV(N) \\
  BV(\lambda x. M) &= \{x\} \cup BV(M)
\end{align*}

Применение абстракции к некоторому аргументу $(\lambda x.M) N$ -- это \textit{подстановка} $M[x \mapsto N]$ терма $N$ вместо \textit{свободных} вхождений переменной $x$ в терме $M$. Формально, правила подстановки:
\begin{align*}
  x[x \mapsto N] &= N \\
  y[x \mapsto N] &= y, (x \neq y) \\
  (T S)[x \mapsto N] &= T[x \mapsto N] S[x \mapsto N] \\
  (\lambda x.T)[x \mapsto N] &= \lambda x.T \\
  (\lambda y.T)[x \mapsto N] &= \lambda y.T[x \mapsto N], (y \notin FV(N), x \neq y)
\end{align*}

Рассмотрим, что произойдет, если в последнем правиле условие $ y \notin FV(N)$ не выполняется:
$$ (\lambda y.x)[x \mapsto y] = \lambda y.y $$

Получилось, что в результате подстановки мы превратили константную функцию $\lambda y.x$ в тождественную. Такая ситуация называется проблемой захвата переменной, когда при подстановке в $\lambda$-абстракцию переменные подставляемого терма захватываются абстракцией.

Эту проблему можно решить если принять так называемое соглашение Барендрегта о том, что имена связанных переменных всегда выбирать так, чтобы они отличались от имен свободных. В примере выше, например, мы можем переименовать связанную переменную $y$ в свежую $z$ и поведении абстракции $\lambda z.x$ не изменится. Тогда подстановку можно использовать без каких-либо оговорок о свободных и связанных переменных. Пример выше превратится в:
$$ (\lambda z.x)[x \mapsto y] = \lambda z.y $$

Как мы уже установили выше, мы можем переименовывать связанные переменные в абстракциях и их поведение при применении к аргументам не изменится. Более того, имена связанных переменных не играют для нас никакой роли. Поэтому, как правило, $\lambda$-термы и рассматривают с точностью до имен параметров абстракций. Формально, на множестве именованных термов $\Lambda$ можно задать отношение $\alphaeq \in \Lambda \times \Lambda$, которое называется $\alpha$-эквивалентностью и определяется как минимальное отношение конгруэнтности, порожденное следующими правилами:

\begin{center}
  \AxiomC{$x \in \mathcal{V}$}
  \UnaryInfC{$x \alphaeq x$}
  \DisplayProof{}
\end{center}

\begin{center}
  \AxiomC{$M \alphaeq M'$}
  \AxiomC{$N \alphaeq N'$}
  \BinaryInfC{$M N \alphaeq M' N'$}
  \DisplayProof{}
\end{center}

\begin{center}
  \AxiomC{$M[x \mapsto y] \alphaeq M'$}
  \AxiomC{$y \notin FV(M)$}
  \BinaryInfC{$\lambda x. M \alphaeq \lambda y.M'$}
  \DisplayProof{}
\end{center}

Наконец, сформулируем лемму о подстановке:

\begin{prop}
  \label{named:assoc}
  Для любых $T, M, N \in \Lambda; x, y \in \mathcal{V}$, если $x \neq y$ и $x \notin FV(M)$, то верно $T[x \mapsto N][y \mapsto M] = T[y \mapsto M][x \mapsto N[y \mapsto M]]$
\end{prop}

\begin{proof}
  Индукция по терму $T$. База индукции -- случай, когда $T$ является переменной. Рассмотрим три случая:
  \begin{enumerate}
    \item $T \equiv x$. Левая часть -- $x[x \mapsto N][y \mapsto M] = N[y \mapsto M]$. Правая часть -- $x[y\mapsto M][x \mapsto N[y \mapsto M]] = N[y \mapsto M]$.
    \item $T \equiv y$. Левая часть -- $y[x \mapsto N][y \mapsto M] = M$. Правая часть -- $y[y\mapsto M][x \mapsto N[y \mapsto M]] = M[x\mapsto N[y \mapsto M]]$. Так как $x \notin FV(M)$, то $M[x\mapsto N[y \mapsto M]] = M$
    \item $T \equiv z \neq x,y$. Обе части редуцируются к $z$.
  \end{enumerate}

  Случай аппликации тривиален, рассмотрим случай абстракции $\lambda z.T$. Рассмотрим возможные случаи:

  \begin{enumerate}
    \item $z \equiv x$. Левая часть -- $(\lambda x.T)[x \mapsto N][y \mapsto M] = (\lambda x.T)[y \mapsto M] \overset{\mathrm{x \notin FV(M)}}{=} \lambda x.T[y \mapsto M]$. Правая часть -- $(\lambda x.T)[y \mapsto M][x \mapsto N[y \mapsto M]] \overset{\mathrm{x \notin FV(M)}}{=} (\lambda x.T[y \mapsto M])[x \mapsto N[y \mapsto M]] = \lambda x.T[y \mapsto M]$.

    \item $z \equiv y$. Пусть $y \in FV(N)$ и $y \in FV(M)$. По соглашению Барендрегта, нам нужно переименовать $y$ в свежую переменную $y'$, такую что $y' \notin FV(N)$ и $y' \notin FV(M)$. Это эквивалентно тому, что мы можем осуществить подстановку $\lambda y'.t[y \mapsto y']$. Вычислим левую часть:
    \begin{gather*}
      (\lambda y'.T[y \mapsto y'])[x \mapsto N][y \mapsto M] = \\
       \lambda y'.T[y \mapsto y'][x \mapsto N][y \mapsto M] \overset{\mathrm{IH}}{=} \\
       \lambda y'.T[y \mapsto y'][y \mapsto M][x \mapsto N[y \mapsto M]]
    \end{gather*}

    Правая часть:
    \begin{gather*}
      (\lambda y'.T[y \mapsto y'])[y \mapsto M][x \mapsto N[y \mapsto M]] = \\
      \lambda y'.T[y \mapsto y'][y \mapsto M][x \mapsto N[y \mapsto M]]
    \end{gather*}

    Заметим здесь, что так как $y' \notin FV(N)$ и $y' \notin FV(M)$, то, очевидно, что $y' \notin FV(N[y \mapsto M])$, поэтому в последнем переходе обе подстановки <<проваливаются>> в абстракцию.

    \item $z \equiv y$. Пусть $y \in FV(N)$, но теперь уже $y \notin FV(M)$. Аналогично предыдущему пункту, мы переименовываем $y$ в свежую переменную $y'$, такую что $y' \notin FV(N)$ и $y' \notin FV(M)$, осуществляя подстановку $\lambda y'.t[y \mapsto y']$. Вычислим левую часть:
    \begin{gather*}
      (\lambda y'.T[y \mapsto y'])[x \mapsto N][y \mapsto M] = \\
       \lambda y'.T[y \mapsto y'][x \mapsto N][y \mapsto M] \overset{\mathrm{IH}}{=} \\
       \lambda y'.T[y \mapsto y'][y \mapsto M][x \mapsto N[y \mapsto M]]
    \end{gather*}

    А в этом случае, заметим, что так как мы сначала заменили \textbf{все} свободные вхождения $y$ в $T$ на $y'$, а потом на место $y$ подставили $M$, то вторая подстановка ничего не делает, поэтому левая часть окончательно равна $\lambda y'.T[y \mapsto y'][x \mapsto N[y \mapsto M]]$

    Правая часть тогда редуцируется в:
    \begin{gather*}
      (\lambda y.T)[y \mapsto M][x \mapsto N[y \mapsto M]] = \\
      (\lambda y.T)[x \mapsto N[y \mapsto M]] = \\
      \lambda y.T[x \mapsto N[y \mapsto M]]
    \end{gather*}

    В последнем шаге вычисления, отметим, что так как $y \in FV(N)$ и $y \notin FV(M)$, то $y \notin FV(N[y \mapsto M])$, поэтому подстановка снова заносится под абстракцию. Переименуем и здесь $y$ в $y'$, тогда получим:
    \begin{gather*}
      \lambda y.T[x \mapsto N[y \mapsto M]] \alphaeq \\
      \lambda y'.T[x \mapsto N[y \mapsto M]][y \mapsto y'] \overset{\mathrm{IH}}{=} \\
      \lambda y'.T[y \mapsto y'][x \mapsto N[y \mapsto M][y \mapsto y']]
    \end{gather*}

    Из тех же соображений, что и выше, подстановка $N[y \mapsto M][y \mapsto y']$ -- это то же самое, что и $N[y \mapsto M]$, поэтому правая часть окончательно вычисляется в $\lambda y'.T[y \mapsto y'][x \mapsto N[y \mapsto M]]$.

    \item $z \neq x,y$. Доказательство аналогично предыдущим двум пунктам. \qedhere

  \end{enumerate}
\end{proof}

\subsection{Неименованное представление термов}
\label{sec:index}
Как уже мы уже видели в предыдущем разделе, имена формальных параметров $\lambda$-абстракций не важны и, в целом, мы можем не обращать на них внимания. Более того, мы можем вообще отказаться от именованных переменных! Широко известен альтернативный способ записи термов через так называемые индексы Де Брауна(De Bruijn) -- \cite{nikolas1972bruijn}. В нем вместо имен переменных используются числовые индексы, показывающие сколько лямбд назад была захвачена переменная. Например комбинатор $S = \lambda f g x. f x (g x)$, записанный в таком представлении будет иметь следующий вид: $\lambda(\lambda(\lambda 3 1 (2 1)))$.

Существует и альтернативный способ такого представления. Множество всех термов разбивается на так называемые <<уровни>>(levels) и вместо него рассматриваются множества $\Lambda_{n}$, где $n$ -- длина контекста, в котором определен терм. О контексте в котором определен терм, можно думать, как о простом списке свободных переменных терма. Индуктивно, они определяются следующим образом:

\begin{center}
  \AxiomC{$0 \leqslant i < n$}
  \UnaryInfC{$v_{n, i} \in \Lambda_{n}$}
  \DisplayProof{}
\end{center}

\begin{center}
  \AxiomC{$T_{1} \in \Lambda_{n}$}
  \AxiomC{$T_{2} \in \Lambda_{n}$}
  \BinaryInfC{$T_{1} T_{2} \in \Lambda_{n}$}
  \DisplayProof{}
\end{center}

\begin{center}
  \AxiomC{$T \in \Lambda_{n + 1}$}
  \UnaryInfC{$\lambda T \in \Lambda_{n}$}
  \DisplayProof{}
\end{center}

В случае переменной индекс $i$ обозначает позицию переменной в контексте. Договоримся отсчитывать ее с конца контекста. Комбинатор $S$, например, в таком представлении будет выглядеть вот так: $\lambda (\lambda (\lambda v_{3,2} v_{3, 0} (v_{3, 1} v_{3, 0})))$.

Такое представление термов удобно потому что $\alpha$-эквивалентность сводится к самому обычному равенству и, как следствие, пропадает проблема коллизии имен переменных.

Определим операцию подстановки для таких термов. Мы определим ее в более общем случае -- вместо какой-то одной переменной мы будем осуществлять подстановку во \textbf{все} переменные терма. Пусть $T \in \Lambda_{n}$, и $S_{0}, \dots S_{n-1} \in \Lambda_{k}$. Тогда $subst(T, S_{n - 1}, \dots, S_{0}) \in \Lambda_{k}$ определяется следующим образом:
\begin{gather*}
  subst(v_{n, i}, S_{n - 1}, \dots, S_{0}) = S_{i} \\
  subst(T, v_{n, n-1}, \dots, v_{n, 0}) = T \\
  subst(T_{1} T_{2}, S_{n - 1}, \dots, S_{0}) = \\
  subst(T_{1}, S_{n - 1}, \dots, S_{0})\ subst(T_{2}, S_{n - 1}, \dots, S_{0}) \\
  subst(\lambda T, S_{n - 1}, \dots, S_{0}) = \lambda (subst(T, w(S_{n - 1}), \dots, w(S_{0}), v_{n+1, 0})
\end{gather*}

Операция $w(T)$ работает следующим образом. Пусть терм $T \in \Lambda_{n}$, тогда терм $w(T) \in \Lambda_{n+1}$ и определен как:
\begin{gather*}
  w(v_{n, i}) = v_{n+1, i+1} \\
  w(T_{1} T_{2}) = w(T_1)\ w(T_2) \\
  w(\lambda T) = \lambda (w(T))
\end{gather*}

Сформулируем и докажем вспомогательную лемму, которая пригодится нам далее:
\begin{lemma}
  \label{index:weak_lemma}
  Пусть $T \in \Lambda_{n}$, а $S_{n-1}, \dots, S_{0} \in \Lambda_{m}$. Тогда $subst(w(T), w(S_{n-1}), \dots, w(S_{0}), v_{m+1, 0}) = w(subst(T, S_{n-1}, \dots S_{0}))$
\end{lemma}

\begin{proof}
  Индукция по структуре терма $T$.
  \begin{enumerate}
    \item База индукции -- $v_{n, i}$. Левая часть равна, по определению подстановки:
    $$ subst(v_{n+1, i+1}, w(S_{n-1}), \dots, w(S_{0}), v_{m+1, 0}) = w(S_{i})$$
    Правая часть:
    $$w(subst(v_{n,i}, S_{n-1}, \dots, S_{0})) = w(S_{i})$$

    \item Случай аппликации снова тривиален. Рассмотрим случай абстракции $\lambda T$. Левая часть вычислится в:
    \begin{gather*}
        subst(w(\lambda T), w(S_{n-1}), \dots, w(S_{0}), v_{m+1, 0}) = \\
        subst(\lambda w(T), w(S_{n-1}), \dots, w(S_{0}), v_{m+1, 0}) = \\
        \lambda subst(w(T), w(w(S_{n-1})), \dots, w(w(S_{0})), v_{m+2, 1}, v_{m+2,0}) \overset{\mathrm{IH}}{=} \\
        \lambda w(subst(T, w(S_{n-1}), \dots, w(S_{0}), v_{m+1, 0}))
    \end{gather*}

    Правая часть вычисляется в:
    \begin{gather*}
        w(subst(\lambda T, S_{n-1}, \dots, S_{0})) = \\
        w(\lambda subst(T, w(S_{n-1}), \dots, w(S_{0}), v_{m+1, 0})) =
        \lambda w(subst(T, w(S_{n-1}), \dots, w(S_{0}), v_{m+1, 0}))
    \end{gather*}
  \end{enumerate}
\end{proof}

Аналогично именованному представлению, сформулируем лемму о подстановке:

\begin{prop}
  \label{index:assoc}
  Пусть $T \in \Lambda_{n}; T_{n - 1}, \dots T_{0} \in \Lambda_{m}, S_{m-1}, \dots S_{0} \in \Lambda_{k}$, тогда верно $subst(subst(T, T_{n - 1}, \dots T_{0}), S_{m-1}, \dots S_{0}) = subst(T, T_{n - 1}' \dots, T_{0}')$, где $T_{i}' = subst(T_{i}, S_{m-1}, \dots S_{0})$.
\end{prop}

Заметим еще, что в таком представлении термов нет необходимости в сторонних условиях, как в лемме о подстановке для именованных термов.

\begin{proof}
  Это утверждение точно так же доказывается индукцией по структуре терма $T$. База индукции -- случай, когда терм представляет собой переменную $v_{n, i}$. Тогда левая часть вычисляется в $subst(T_{i}, S_{m-1}, \dots S_{0})$, ровно как и правая.

  Случай аппликации снова тривиален, рассмотрим случай абстракции $\lambda T$. Вычислим левую часть:
  \begin{gather*}
    subst(subst(\lambda T, T_{n-1}, \dots, T_{0}), S_{m-1}, \dots, S_{0}) = \\
    subst(\lambda subst( T, w(T_{n - 1}), \dots w(T_{0}), v_{m+1, 0} ), S_{m - 1}, \dots, S_{0}) = \\
    \lambda(subst(subst( T, w(T_{n - 1}), \dots w(T_{0}), v_{m+1, 0} ), w(S_{m-1}), \dots, w(S_{0}), v_{k+1, 0}) \overset{\mathrm{IH}}{=} \\
    \lambda(subst(T, subst(w(T_{n-1}), w(S_{m-1}), \dots, w(S_{0}), v_{k+1, 0}), \dots, \\
    subst(w(T_{0}), w(S_{m-1}), \dots, w(S_{0}), v_{k+1, 0}), subst(v_{m+1, 0}, w(S_{m-1}), \dots, w(S_{0}), v_{k+1, 0}))) = \\
    \lambda(subst(T, subst(w(T_{n-1}), w(S_{m-1}), \dots, w(S_{0}), v_{k+1, 0}), \dots, \\
    subst(w(T_{0}), w(S_{m-1}), \dots, w(S_{0}), v_{k+1, 0}), v_{k+1, 0}))
  \end{gather*}

  Вычислим теперь правую часть:
  \begin{gather*}
    subst(\lambda T, subst(T_{n-1}, S_{m - 1}, \dots, S_{0}), \dots, subst(T_{0}, S_{m - 1}, \dots, S_{0})) = \\
    \lambda(subst(T, w(subst(T_{n-1}, S_{m - 1}, \dots, S_{0})), \dots, w(subst(T_{0}, S_{m - 1}, \dots, S_{0})), v_{k+1, 0}))
  \end{gather*}

  По лемме~\ref{index:weak_lemma} $subst(w(T_{i}), w(S_{m-1}), \dots, w(S_{0}), v_{k+1, 0}) = w(subst(T_{i}, S_{m - 1}, \dots, S_{0}))$ для всех $i=\overline{0, n-1}$, следовательно наше утверждение верно.
\end{proof}

\subsection{Монадическое представление термов}
\label{sec:monad}
Существует еще один способ записи $\lambda$-термов, описанный в \cite{bird1999bruijn}. Библиотека \textbf{Bound}~\cite{bound} для языка \textbf{Haskell}, например, использует именно монадическое представление.

Основная идея в том, что именованное представление для термов можно обобщить и свободные переменные брать из произвольного множества $V$. Тогда множество термов $\Lambda_{V}$ определяется индуктивно по следующим правилам:
\begin{center}
  \AxiomC{$v \in V$}
  \UnaryInfC{$v \in \Lambda_{V}$}
  \DisplayProof{}
\end{center}

\begin{center}
  \AxiomC{$M \in \Lambda_{V}$}
  \AxiomC{$N \in \Lambda_{V}$}
  \BinaryInfC{$M N \in \Lambda_{V}$}
  \DisplayProof{}
\end{center}

\begin{center}
  \AxiomC{$M \in \Lambda_{V \coprod \{*\}}$}
  \UnaryInfC{$\lambda M \in \Lambda_{V}$}
  \DisplayProof{}
\end{center}

Здесь $\{*\}$ -- это произвольное одноэлементное множество, а $\coprod$ -- операция размеченного объединения множеств. По определению $A \coprod B$ состоит из элементов $inl(a)$ и $inr(b)$, где $a \in A$ и $b \in B$.  Так как для абстракции нам нужно иметь на одну свободную переменную больше, то ее можно получить взяв размеченное объединение с произвольным одноэлементным множеством. Это представление так же удобно для компьютерной реализации за счет того, что проверку корректности построения термов можно выполнять на уровне типов.

Пусть у нас есть функция $f : V \to W$, тогда мы можем задать функцию $F_{f}$ из $\Lambda_{V}$ в $\Lambda_{W}$ рекурсией по терму $T \in \Lambda_{V}$:

\begin{enumerate}
  \item $v \mapsto f(v)$
  \item $M\ N \mapsto F_{f}(M)\ F_{f}(N)$
  \item $\lambda M \mapsto \lambda F_{f'(f)}(M)$. Заметим, что просто так отобразить терм $M$ с помощью функции $f$ мы не можем, так как ее домен не совпадает со множеством, которым параметризован тип терма $M$. Поэтому мы построим по $f$ функцию $f'(f) : V \coprod \{*\} \to W \coprod \{*\}$. Устроена она будет следующим образом:
  \begin{enumerate}
    \item $f'(f)(inl(x)) = inl(f(x))$
    \item $f'(f)(inr(*)) = inr(*)$
  \end{enumerate}
\end{enumerate}

Знакомый с языком программирования \textbf{Haskell} или теорией категорий читатель узнает, что мы задали структуру функтора. Интуитивно, действие этого функтора -- это переименование переменных. Покажем, что это действительно функтор, именно, что он уважает тождественное отображение и композицию отображений.

\begin{prop}
  \label{monad:fmap-resp-id}
  Для любого $T \in \Lambda_{V}$ верно, что $F_{id_{V}}(T) = T$
\end{prop}

\begin{proof}
  Индукция по терму $T$. База тривиальна, равно как и случай аппликации, покажем, что утверждение верно и для случая лямбды. Вспомогательная функция $f'(f)$ устроена следующим образом:
  \begin{enumerate}
    \item $f'(id_{V})(inl(x)) = inl(x)$
    \item $f'(id_{V})(inr(*)) = inr(x)$
  \end{enumerate}
  Следовательно, оно является тождеством на $V \coprod \{*\}$. По индукционной гипотезе получаем, что случай для лямбды тоже верен.

  Формальное доказательство этого утверждения можно увидеть в приложении~\ref{apendix:monad}, в функции \texttt{fmap-respects-id}
\end{proof}

\begin{prop}
  \label{monad:fmap-resp-comp}
  Для любого $T \in \Lambda_{V}$ и $f : V \to W$, $g : W \to X$ верно, что $F_{g \circ f}(T) = F_{g}(T) \circ F_{f}(T)$
\end{prop}

\begin{proof}
  Снова индукция по терму $T$. Случай переменной снова тривиален, случай аппликации напрямую следует из индукционной гипотезы, рассмотрим случай абстракции. Посмотрим, во что вычислится левая часть:
  \begin{gather*}
    F_{g \circ f}(\lambda M) = \lambda F_{f'(g \circ f)}(M) \\
    \text{где} \\
    f'(g \circ f)(inl(v)) = inl(g(f(v))) \\
    f'(g \circ f)(inr(*)) = inr(*)
  \end{gather*}

  Правая, в свою очередь:
  \begin{gather*}
    F_{g}(F_{f}(\lambda M)) = F_{g}(\lambda F_{f'(f)}(M)) = \\
    \lambda F_{g'(g)}(F_{f'(f)}(M))
  \end{gather*}

  Вспомогательная функция $f'(f) : V \coprod \{*\} \to W \coprod \{*\}$ устроена здесь следующим образом:
  \begin{enumerate}
    \item $f'(f)(inl(v)) = inl(f(v))$
    \item $f'(f)(inr(*)) = inr(*)$
  \end{enumerate}

  Вспомогательная функция $g'(g) : W \coprod \{*\} \to X \coprod \{*\}$ устроена здесь следующим образом:
  \begin{enumerate}
    \item $g'(g)(inl(w)) = g'(g)(inl(f(v))) = inl(g(f(v)))$
    \item $g'(g)(inr(*)) = inr(*)$
  \end{enumerate}

  Несложно увидеть, что эти две функции из левой и правой частей ведут себя одинаково, следовательно и для случая лямбды утверждение верно.
  Формальное доказательство этого утверждения можно увидеть в приложении~\ref{apendix:monad}, в функции \texttt{fmap-respects-сomp}
\end{proof}

Мы пойдем еще дальше и зададим структуру монады. Существует много способов определить её, мы воспользуемся тем, который принят в языке программирования \textbf{Haskell}. Именно, мы определим две операции: монадическую единицу $\texttt{return} : V \to \Lambda_{V}$ и монадическое связывание $\texttt{bind} : \Lambda_{V} \to (V \to \Lambda_{W}) \to \Lambda_{W}$. Кроме этого, мы покажем, что они удовлетворяют монадическим законам.

Монадическая единица устроена очень просто -- это переменная. Связывание же, принимает на вход так называемую стрелку Клейсли~\cite{kleisliArrows} $k : V \to \Lambda_{W}$ и определяется индуктивно по структуре терма $T$:

\begin{enumerate}
  \item $v \mapsto k(v)$
  \item $M\ N \mapsto (\texttt{bind}(M, k))\ (\texttt{bind}(N, k))$
  \item $\lambda M \mapsto \lambda(\texttt{bind}(M, k'(k)))$, где $k'(k) : V \coprod \{*\} \to \Lambda_{W \coprod \{*\}}$ и определяется следующим образом:
    \begin{enumerate}
      \item $k'(k)(inl(v)) = F_{inl}(k(v))$
      \item $k'(k)(inr(*)) = \texttt{return}(inr(*))$
    \end{enumerate}
\end{enumerate}


Сформулируем и докажем теперь свойства этих двух операций.

\begin{prop}
  \label{monad:bind-right-unit}
  Для любого $T \in \Lambda_{V}$ верно $\texttt{bind}(T, \texttt{return}) = T$.
\end{prop}

\begin{proof}
  Индукция по структуре терма $T$:
  \begin{enumerate}
    \item База индукции, $T = v$. Имеем, что $\texttt{bind}(v, \texttt{return}) = \texttt{return}(v) = v$.
    \item Случай аппликации напрямую следует из предположения индукции.
    \item Пусть теперь $T = \lambda M$. Имеем, что $\texttt{bind}(\lambda M, \texttt{return}) = \lambda (\texttt{bind}(M, k'(\texttt{return})))$. Вспомогательное отображение $k'(return)$ устроено следующим образом:
    \begin{enumerate}
      \item $k'(\texttt{return})(inl(v)) = F_{inl}(\texttt{return}(v)) = \texttt{return}(inl(v))$
      \item $k'(\texttt{return})(inr(*)) = \texttt{return}(inr(*))$
    \end{enumerate}
  \end{enumerate}

  Заметим, что оно ведет себя так же, как и \texttt{return} на $V \coprod \{*\}$, следовательно по индукционной гипотезе получаем исходное утверждение.
  Формальное доказательство этого утверждения можно увидеть в приложении~\ref{apendix:monad}, в функции \texttt{bind-right-unit}.
\end{proof}

\begin{prop}
  \label{monad:bind-left-unit}
  Для любого $v \in V$ и $k : V \to \Lambda_{V}$ верно $\texttt{bind}(\texttt{return}(v), k) = k(v)$
\end{prop}

\begin{proof}
  Утверждение тривиально следует из определения \texttt{bind} и того факта, что \texttt{return} -- это переменная. Формальное доказательство этого утверждения можно увидеть в приложении~\ref{apendix:monad}, в функции \texttt{bind-left-unit}.
\end{proof}

Прежде, чем формулировать последний монадный закон, сформулируем и докажем несколько технических лемм, которые помогут нам в его доказательстве.

\begin{lemma}
  \label{monad:bind-fmap-comm-lhs}
  Для любых $T \in \Lambda_{V}$, $f : V \to W$ и $k : W \to \Lambda_{U}$ верно $\texttt{bind}(F_{f}(T), k) = \texttt{bind}(T, k \circ f)$.
\end{lemma}

\begin{proof}
  Индукция по структуре терма $T$ с тривиальными случаями переменной и аппликации. Рассмотрим случай абстракции $\lambda M$. Нам нужно показать, что $\lambda \texttt{bind}(F_{f'(f)}(M), k'(k)) = \lambda \texttt{bind}(M, k'(k \circ f))$. По индукционной гипотезе мы знаем, что $\texttt{bind}(F_{f'(f)}(M), k'(k)) = \texttt{bind}(M, k'(k) \circ f'(f))$, Заметим теперь, что $k'(k \circ f)$ и $k'(k) \circ f'(f)$ ведут себя одинаково на всех входах, следовательно это утверждение доказано. Формальное доказательство этого утверждения можно увидеть в приложении~\ref{apendix:monad}, в функции \texttt{bind-fmap-comm-lhs}.
\end{proof}

\begin{lemma}
  \label{monad:bind-fmap-comm-rhs}
  Для любых $T \in \Lambda_{V}$, $f : W \to U$ и $k : V \to \Lambda_{W}$ верно $ F_{f}(\texttt{bind}(T, k)) = \texttt{bind}(T, (x \mapsto F_{f}(k(x))))$.
\end{lemma}

\begin{proof}
  Снова индукция по структуре терма $T$. Случай переменной и аппликации снова тривиален, поэтому рассмотрим случай абстракции $\lambda M$.

  Нам нужно показать, что $\lambda F_{f'(f)}(\texttt{bind}(M, k'(k))) = \lambda \texttt{bind}(M, k'(x \mapsto F_{f}(k(x))))$. По индукционной гипотезе мы знаем, что $F_{f'(f)}(\texttt{bind}(M, k'(k))) = \texttt{bind}(M, (x \mapsto F_{f'(f)}(k'(k)(x))))$. Покажем теперь, что стрелки Клейсли $ k'(x \mapsto F_{f}(k(x))) $ и $ x \mapsto F_{f'(f)}(k'(k)(x)) $ ведут себя одинаково на всех входах:

  \begin{enumerate}
    \item $inr(*)$. Обе стрелки вычисляются в $\texttt{return}(inr(*))$
    \item $inl(v)$. Надо показать, что $F_{f'(inl)}(F_{f}(k(v))) = F_{f'(f)}(F_{inl}(k(v)))$. Так как $\Lambda_{-}$ -- функтор и он уважает композицию отображений имеем, что нужно показать $F_{f'(inl) \circ f}(k(v)) = F_{f'(f) \circ inl}(k(v))$. Для этого в свою очередь нужно снова показать, что два отображения $f'(inl) \circ f$ и $f'(f) \circ inl$ ведут себя одинаково на всех входах, но это очень легко понять, просто взглянув на определение $f'$.
  \end{enumerate}

  Формальное доказательство этого утверждения можно увидеть в приложении~\ref{apendix:monad}, в функции \texttt{bind-fmap-comm-rhs}.
\end{proof}

\begin{lemma}
  \label{monad:bind-fmap-comm}
  Для любого $T \in \Lambda_{V}$ и $f : V \to \Lambda_{W}$ верно $\texttt{bind}(F_{inl}(t), k'(g)) = F_{inl}(\texttt{bind}(T, f))$.
\end{lemma}

\begin{proof}
  Снова индукция по структуре терма $T$, случай переменной и аппликации тривиален, рассмотрим случай абстракции $\lambda M$.

  Левая часть вычислится в:
  $$ \lambda F_{f'(inl)}(\texttt{bind}(M, k'(f))) $$
  Правая в:
  $$ \lambda \texttt{bind}(F_{f'(inl)}(M), k'(k'(f))) $$

  По лемме~\ref{monad:bind-fmap-comm-lhs} имеем, что $\texttt{bind}(F_{f'(inl)}(M), k'(k'(f))) = \texttt{bind}(M, k'(k'(f)) \circ f'(inl))$. По лемме~\ref{monad:bind-fmap-comm-rhs} имеем $F_{f'(inl)}(\texttt{bind}(M, k'(f))) = \texttt{bind}(M, (x \mapsto F_{f'(inl)}(k'(f)(x))))$. Заметим теперь, что стрелки Клейсли $k'(k'(f)) \circ f'(inl)$ и $x \mapsto F_{f'(inl)}(k'(f)(x))$ ведут себя одинаково на всех входах, тогда по симметричности и транзитивности равенства получаем доказательство требуемого утверждения.

  Формальное доказательство этого утверждения можно увидеть в приложении~\ref{apendix:monad}, в функции \texttt{bind-fmap-comm}.
\end{proof}

\begin{prop}
  \label{monad:bind-assoc}
  Для любого $T \in \Lambda_{V}$, $f : V \to \Lambda_{W}, g : W \to \Lambda_{U}$ верно $\texttt{bind}(\texttt{bind}(T, f), g) = \texttt{bind}(T, (x \mapsto \texttt{bind}(f(x), g)) )$.
\end{prop}

\begin{proof}
  Индукция по терму $T$:
  \begin{enumerate}
    \item Рассмотрим случай, когда $T = v$. Тогда левая часть вычисляется в $\texttt{bind}(f(v), g)$, ровно как и правая.
    \item Случай для аппликации следует напрямую из индукционной гипотезы.
    \item Рассмотрим случай абстракции $\lambda M$. Посмотрим, во что вычисляется левая часть:
    \begin{gather*}
      \texttt{bind}(\texttt{bind}(\lambda M, f), g) = \texttt{bind}(\lambda \texttt{bind}(M, k'(f)), g) = \\
      \lambda \texttt{bind}(\texttt{bind}(M, k'(f)), k'(g)) \\
      \text{где} \\
      k'(g)(inl(w)) = F_{inl}(g(w)) \\
      k'(g)(inr(*)) = \texttt{return}(inr(*)) \\
      \text{и} \\
      k'(f)(inl(v)) = F_{inl}(f(v)) \\
      k'(f)(inr(*)) = \texttt{return}(inr(*))
    \end{gather*}

    Правая часть, в свою очередь, вычисляется в:
    \begin{gather*}
      \texttt{bind}(\lambda M, (x \mapsto \texttt{bind}(f(x), g))) = \lambda \texttt{bind}(M, k'(x \mapsto \texttt{bind}(f(x), g)))
    \end{gather*}

    Чтобы воспользоваться индукционной гипотезой, необходимо показать, что $k'(x \mapsto \texttt{bind}(f(x), g)) : V \coprod \{*\} \to \Lambda_{U \coprod \{*\}}$ ведет себя так же как и $x \mapsto \texttt{bind}(k'(f)(x), k'(g))$. Для этого мы просто покажем, что они возвращают одинаковый результат на всех входах.

    Рассмотрим два случая, как могут выглядеть входные данные:
    \begin{enumerate}
      \item $inr(*)$. Обе части вычисляются в $\texttt{return}(inr(*))$.
      \item $inl(v)$. Левая часть вычисляется в: $$ F_{inl}(\texttt{bind}(f(v), g)) $$
      Правая: $$\texttt{bind}(F_{inl}(f(v)), f'(g))$$
      Воспользовавшись леммой~\ref{monad:bind-fmap-comm} для терма $f(v)$ и $g$ получаем доказательство исходного утверждения. \qedhere
    \end{enumerate}
  \end{enumerate}

  Формальное доказательство этого утверждения можно увидеть в приложении~\ref{apendix:monad}, в функции \texttt{bind-assoc}.
\end{proof}


Отметим наконец, что действие функтора мы проинтерпретировали, как переименование переменных. Действие монадического связывания можно, в таком случае, проинтерпретировать как подстановку. Это утверждение не столь очевидно, но если рассмотреть сигнатуру \texttt{bind} и обратить внимание на то, что функцию $k : V \to \Lambda_{W}$ можно задать в виде списка пар $(V, \Lambda_{W})$, то это соответствие становится куда более явным. Монадные законы, в свою очередь, в точности описывают свойства подстановки, которые мы в явном виде задали в прошлых разделах~\ref{sec:named} и \ref{sec:index}.

\subsection{Преобразования между представлениями}
\label{sec:conversions}
В этом разделе мы опишем преобразования между представлениями и начнем с преобразования именованных термов в неименованные. Очевидно, что для осуществления этого нам необходимо знать порядок на переменных в терме. Поэтому мы считаем, что кроме самого терма нам дают контекст.

\begin{definition}
  \textbf{Контекст} $\Gamma$ -- это не содержащий дубликатов список переменных $x_{1}, \dots x_{n}$, $x_{i} \in \mathcal{V}$.
\end{definition}

\begin{definition}
  Терм $T$ определен в контексте $\Gamma$ тогда и только тогда, когда все свободные переменные терма $T$ присутствуют в $\Gamma$. Этот факт традиционно обозначается как $\Gamma \vdash T$.
\end{definition}

Итак, преобразование $\Phi$ именованных термов в неименованные принимает на вход контекст $\Gamma = x_{1}, \dots, x_{n}$, терм $T \in \Lambda$  такой что он определен в контексте $\Gamma$ и возвращает неименованный терм $T' \in \Lambda_{n}$. Определяется оно индукцией по структуре терма $T$:

\begin{enumerate}
  \item $x_{1}, \dots, x_{n} \vdash x_{i} \mapsto v_{n, n - i}$
  \item $x_{1}, \dots, x_{n} \vdash M N \mapsto \Phi(x_{1}, \dots, x_{n} \vdash M)\ \Phi(x_{1}, \dots, x_{n} \vdash N)$
  \item $x_{1}, \dots, x_{n} \vdash \lambda x.M \mapsto \lambda \Phi(x_{1}, \dots, x_{n} , x \vdash M)$
\end{enumerate}

Покажем, что такое преобразование уважает отношение $\alpha$-эквивалентности, введенное в разделе~\ref{sec:named}.

\begin{prop}
  Пусть $T_{1}, T_{2} \in \Lambda$, $\Gamma \vdash T_{1}$, $\Gamma \vdash T_{2}$ и $T_{1} \alphaeq T_{2}$. Тогда $\Phi(\Gamma \vdash T_{1}) = \Phi(\Gamma \vdash T_{2})$.
\end{prop}

\begin{proof}
  Одновременная индукция по структуре термов $T_{1}$ и $T_{2}$. Так как мы знаем, что они $\alpha$-эквивалентны, то нам нет необходимости рассматривать всевозможные комбинации термов. Поэтому рассмотрим лишь случаи, когда термы имеют общую структуру(две переменные, две аппликации или две абстракции).

  \begin{enumerate}
    \item База индукции. $T_{1} = x$, $T_{2} = y$, так как они альфа-эквивалентны, то $x \equiv y$, а так как они определены в одинаковом контексте, то и стоят на одинаковых позициях, по определению $\Phi$, получаем, что база индукции верна.
    \item Рассмотрим случай, когда $T_{1} = \lambda x.M$, $T_{2} = \lambda y.M'$. Так как они $\alpha$-эквивалентны, то мы знаем, что $M' \alphaeq M[x \mapsto y]$ и $y \notin FV(M)$. Мы знаем, что $\Gamma \vdash \lambda x.M$ и $\Gamma \vdash \lambda y.M'$, следовательно $\Gamma, x \vdash M$ и $\Gamma, y \vdash M'$. Кроме того, так как $y \notin FV(M)$, то $\Gamma, y \vdash M[x \mapsto y]$.
    По посылке из правила альфа-эквивалентности для абстракций мы знаем, что $M' \alphaeq M[x \mapsto y]$, следовательно по индукционной гипотезе $\Phi(\Gamma, y \vdash M') = \Phi(\Gamma, y \vdash M[x \mapsto y])$. Осталось заметить, что $\Phi(\Gamma, y \vdash M[x \mapsto y]) = \Phi(\Gamma, x \vdash M)$ и получить доказательство исходного утверждения. \qedhere
  \end{enumerate}
\end{proof}

Сконструируем теперь преобразование в обратную сторону -- из неименованных термов в именованные. Нам хотелось бы, что бы оно принимало на вход терм $T' \in \Lambda_{n}$ и возвращало пару из контекста $\Gamma$ и терма $T \in \Lambda$, определенного в нем. Рассмотрим три случая, как бы мы могли задать это преобразование, назовем его $\Psi$.

\begin{enumerate}
  \item В случае, когда $T' = v_{n, i}$ мы можем просто cгенерировать $n$ именованных переменных $\Gamma = x_{1}, \dots x_{n}$ и в качестве результата вернуть пару из контекста $\Gamma$ и $x_{n - i}$. То есть:
  $$ v_{n, i} \mapsto x_{1}, \dots x_{n} \vdash x_{n-i} $$

  \item В случае аппликации $M\ N$ кажется, что все еще проще. Так как она определена в том же контексте, что и оба аппликанта, то нам достаточно пары рекурсивных вызовов и мы можем взять любой контекст в окончательный результат. То есть:
  $$ M\ N \mapsto \pi_{1}(\Psi(M)) \vdash \pi_{2}(\Psi(M))\ \pi_{2}(\Psi(N))$$
  Здесь и далее $\pi_{1}$ и $\pi_{2}$ -- первая и вторая проекция для пар соответственно.
  Но здесь нас поджидает неприятный момент, связанный с тем, что рекурсивные вызовы, возвращают нам два \textit{различных} контекста, в которых определены оба аппликанта. Поэтому формально, мы не можем так определить преобразование из неименованных термов в именованные.

  % \item Для абстракции $\lambda T$ действуем примерно так же. Вызываемся рекурсивно от $T$ и получаем терм, который определен в расширенном на одну переменную контексте. Так как контекст расширяется путем добавления переменной в конец, то мы точно знаем, по какой переменной абстрагироваться:
  % $$ \lambda T \mapsto take(n, \pi_{1}(\Psi(T))) \vdash \lambda\ last(\pi_{1}(\Psi(T)))\ \pi_{2}(\Psi(T)) $$
  % Здесь $take(n, xs)$ -- операция, берущая первые $n$ элементов из списка $xs$, а $last(xs)$ -- операция, возвращающая последний элемент в списке.
\end{enumerate}

Для того, что бы корректно определить $\Psi$, заметим, что нам вообще-то не важно, в каком контексте будет определен результирующий терм. Мы знаем его длину, следовательно, мы можем сгенерировать его и подать на вход обратному преобразованию. Тогда оно вернет нам именованный терм, определенный в данном контексте. Корректное определение $\Psi$, выглядит следующим образом:

\begin{enumerate}
  \item $\Gamma, v_{n, i} \mapsto \Gamma_{n - i}$
  \item $\Gamma, M N \mapsto \Psi(\Gamma, M)\ \Psi(\Gamma, N)$
  \item $\Gamma, \lambda M \mapsto \lambda x' \Psi(\Gamma; x', M)$
\end{enumerate}

В последнем случае, переменная $x'$ выбирается <<свежей>>, в том смысле, что и в разделе~\ref{sec:named}. Запись $\Gamma; x'$ обозначает расширение контекста $\Gamma$, путем дописывания в его конец, переменной $x'$.

Легко заметить, что эти два преобразования взаимно-обратны. Однако необходимо оговориться, что мы установили биекцию между множеством пар из именованных термов и их контекстов.

Сконструируем преобразования между неименованными и монадическими термами. Для примера покажем преобразование $\Delta$ из $\Lambda_{n}$ в $\Lambda_{\overline{n}}$, где $\overline{n}$ -- это множество $\{0,1,\dots, n-1\}$. Действительно, рассмотрим три случая:

\begin{enumerate}
  \item $v_{n, i}$. Так как $0 \leqslant i < n$, то $i$ и так лежит в $\Lambda_{\overline{n}}$
  $$ v_{n,i} \mapsto i $$
  \item $M\ N$. Вызываемся рекурсивно от обеих частей:
    $$ M\ N \mapsto \Delta(M)\ \Delta(N) $$
  \item $\lambda M$. В этом случае придется воспользоваться тем, что $\Lambda_{\overline{n+1}}$ является функтором. Сначала вызовемся рекурсивно от $M$ и получим терм, лежащий в $\Lambda_{\overline{n+1}}$. А затем отобразим его в $\Lambda_{\overline{n} \coprod \{*\}}$ следующим образом:
  \begin{gather*}
    f : \overline{n+1} \to \overline{n} \coprod \{*\}\\
    0 \mapsto inr(*) \\
    i \mapsto inl(i-1)
  \end{gather*}
  Окончательно:
  $$ \lambda M \mapsto \lambda F_{f}(\Delta(M))  $$
\end{enumerate}

Аналогичным образом конструируется преобразование в обратную сторону $\Theta : \Lambda_{\overline{n}} \to \Lambda_{n}$:

\begin{enumerate}
  \item $ i \mapsto v_{n, i} $
  \item $ M\ N \mapsto \Theta(M)\ \Theta(N) $
  \item $\lambda M \mapsto \lambda \Theta(F_{f^{-1}}(M))$
  где:
  \begin{gather*}
    f^{-1} : \overline{n} \coprod \{*\} \to \overline{n + 1} \\
    inl(i) \mapsto i + 1 \\
    inr(*) \mapsto 0
  \end{gather*}
\end{enumerate}

Эти преобразования также взаимно-обратны, мы не будем приводить доказательство этого факта, полагая его очевидным.

